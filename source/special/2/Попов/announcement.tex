% $date: 6--9 июня 2018 г.

\worksheet*{Многочлены в~арифметике остатков}

% $authors:
% - Леонид Андреевич Попов

% $style[-announcement]:
% - .[announcement]

Рассмотрим многочлен вида
\[
    a_{n} x^{n} + a_{n-1} x^{n-1} + \ldots + a_{1} x + a_{0}
\, , \]
коэффициенты $a_{k}$ и~переменная $x$ в~котором~--- это вычеты по~некоторому
натуральному модулю~$m$.
Тогда его значение тоже будет определено по~модулю~$m$.

На~спецкурсе мы вспомним некоторые утвержения теории многочленов для
действительных чисел (решение квадратных уравнений, теорему Виета) и~их
доказательства;
а~потом поймем, как они переносятся на~многочлены по~натуральному модулю.

Чтобы решить квадратные уравнения, нам придется познакомится с~понятиями
квадратичного вычета и~невычета, а~также с~квадратичным законом взаимности.

Никаких предварительных требований к~слушателям нет.
Зачет будет ставится по~итогам сдачи задач.


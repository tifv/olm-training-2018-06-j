% $date: 6--9 июня 2018 г.

\worksheet*{От~неравенств и~оценок до~постулата Бертрана}

% $authors:
% - Сергей Александрович Дориченко

% $style[-announcement]:
% - .[announcement]

Кто кого <<перегонит>> при больших $n$, если <<соревнуются>>, к~примеру,
$2^{n}$ и~$n^{100}$?
Всегда~ли найдется простое число между $n$ и~$2n$?
Как неравенства и~оценки помогают в~задачах, где в~условии нет никаких
неравенств.


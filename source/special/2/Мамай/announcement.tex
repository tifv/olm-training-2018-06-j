% $date: 6--9 июня 2018 г.

\worksheet*{Сложность алгоритмов}

% $authors:
% - Игорь Борисович Мамай

% $style[-announcement]:
% - .[announcement]

% $build$style[print]:
% - .[a5paper,resize-to]

Данный спецкурс посвящен теме эффективных алгоритмов и~их сложности.
Будут затронуты алгоритмы, связанные с~задачами теории чисел, а~также
алгоритмы, решающие комбинаторные задачи.
Спецкурс не~требует специальной подготовки.

В~первый день будут даны основы модульной арифметики, будет изложен метод
нахождения обратного элемента по~простому модулю на~основе малой теоремы Ферма,
а~также алгоритм быстрого возведения в~степень.
На~примере быстрого возведения в~степень будет рассказано о~понятии
вычислительной сложности алгоритма.
Будет рассмотрена задача нахождения $n$-го числа Фибоначчи, рассказано
о~проблемах связанных с~использованием явной формулы и~о~методе, использующем
матрицу перехода, который позволяет находить остаток от~деления $n$-го числа
Фибоначчи по~заданному модулю для огромных значений $n$.

Во~второй день будет изложен алгоритм, позволяющий эффективно находить все
делители заданного целого числа, а~также раскладывать это число на~простые
множители.
Далее будет разобран алгоритм <<Решето Эратосфена>> и~его модификации, а~также
способы, позволяющие дополнительно вычислять различные арифметические функции,
например, функцию Эйлера, функцию Мебиуса, количество делителей числа, сумму
делителей числа.
Кратко будет рассказано об~алгоритмах проверки числа на~простоту.

В~третий день речь пойдет о~таком способе решения задач, как
<<Динамическое программирование>>.
Будет рассказано о~различных видах динамического программирования на~примере
классических задач по~этой теме.

В~четвертый день будет разобран класс задач на~динамическое программирование
по~профилю.
Затем будет проведен зачет по~пройденным темам.


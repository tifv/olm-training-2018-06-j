% $date: 10--13 июня 2018 г.

\worksheet*{Фонология: преобразования звуков в~языке}

% $authors:
% - Александр Чедович Пиперски

% $style[-announcement]:
% - .[announcement]

Формы русского слова <<голова>> звучат так:
[галава́], [галавы́], [галав’е́], [го́лаву], [галаво́й], [галав’е́],
[го́лавы], [гало́ф], [галава́м], [галава́м'и], [галава́х].
Корень этого слова может звучать четырьмя разными способами:
[галав], [галав’], [го́лав], [гало́ф],~---
но~ни~в~одном словаре вы не~найдёте такого перечисления.

Дело в~том, что мы склонны считать, что эти варианты получаются из~единой формы
<<голов>> с~помощью простых преобразований:
безударное <<о>> превращается в~[а], а~<<в>> переходит в~[ф] в~конце слова
и~смягчается перед <<е>>.
Но~такие преобразования могут быть гораздо более сложными, и~тогда надо понять,
как они взаимодействуют друг с~другом.
Например, существует теория, которая представляет русское слово
<<переодевающаяся>> как
\[ \text{per=o=dē+w+ō+ī+o+nt+j+ō-i+ō-sim}, \]
а~чтобы получить его реальное звучание, надо применить длинную цепочку
упорядоченных правил.
О~том, как устроены такие правила и~цепочки правил, которые управляют русским
языком и~другими языками мира, мы и~поговорим на~спецкурсе.


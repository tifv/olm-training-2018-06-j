% $date: 10--13 июня 2018 г.

\worksheet*{Бесконечные множества и~аксиома выбора}

% $authors:
% - Григорий Алексеевич Юргин

% $style[-announcement]:
% - .[announcement]

Первая половина курса будет посвящена понятию мощности бесконечного множества.
Мы выясним, почему точек на~отрезке длины~1 больше, чем натуральных чисел,
но~при этом столько~же, сколько точек на~отрезке длины 2 и~столько~же, сколько
точек внутри квадрата.

Во~второй половине курса будут сформулированы аксиома выбора и~лемма Цорна
(которые, к~слову, эквивалентны друг другу).
Они позволят доказывать существование каких-либо объектов, не~строя эти объекты
явно.
Мы обсудим разные следствия аксиомы выбора, некоторые из~которых противоречат
нашей интуиции.
Например, мы докажем существование множества, не~имеющего площади.
Еще мы рассмотрим функции~$f$, удовлетворяющие условию $f(x + y) = f(x) + f(y)$
при всех действительных $x, y$.
Понятно, что линейные функции $y = k x$ этому удовлетворяют.
Но~есть~ли другие такие функции?\ldots


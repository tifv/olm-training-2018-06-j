% $date: 10--13 июня 2018 г.

\worksheet*{Многочлены и~кривые}

% $authors:
% - Сергей Александрович Дориченко

% $style[-announcement]:
% - .[announcement]

Мы начнём с~многочленов от~одной переменной, а~потом перейдём к~многочленам
от~двух переменных и~кривым на~плоскости, которые задаются этими многочленами.

Мы дадим красивое доказательство теорем Паппа и~Паскаля с~помощью многочленов.

Среди вопросов, которые будут затрагиваться: можно~ли задать на~плоскости
многочленом ветвь гиперболы, в~скольких точках могут пересекаться две
алгебраические кривые на~плоскости и~др.

Приглашаются 9-классники (надо знать уравнение окружности, параболы, прямой
в~декартовых координатах).


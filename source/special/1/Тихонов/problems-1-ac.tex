% $date: 2--5 июня 2018 г.

\worksheet*{Набор математических текстов в~{\LaTeX}. Задачи-1}

% $authors:
% - Юлий Васильевич Тихонов

% $build$matter[print]: [[.], [.]]

\emph{Можно брать понравившиеся вам задачи из~листочков (но~слишком простые я
не~буду засчитывать).}


\subsection*{Алгебра}

\begin{problems}

\itemx{$^8$}
Найдите НОД всех чисел, в~записи каждого из~которых все цифры
$1, 2, 3, \ldots, 9$ использованы по~одному разу.

\item
При каких положительных значениях $x$ дробь $\frac{81 + 16 x^4}{x^2}$ принимает
наименьшее значение?

\item
Докажите, что $(x + y) (y + z) (x + z) \geq 8 x y z$.

\item
Положим величину $T_{n}$ равной
\(
    \left(1 + \frac{1}{1}\right) \left(1 + \frac{1}{3}\right)
    \ldots
    \left(1 + \frac{1}{2n - 1}\right)
\).
Докажите, что $T_{n} \geq \sqrt{2n + 1}$.

\item
Может~ли оканчиваться на~$3$ сумма делителей числа $n$ (считая единицу и~само
число), оканчивающегося на~$3$?

\end{problems}


\subsection*{Комбинаторика}

\begin{problems}

\itemx{$^8$}
Докажите, что за~$4$~взвешивания нельзя среди $100$ одинаковых монет найти
одну фальшивую, которая легче настоящих.

\item
Имеется 20~человек~--- 10~юношей и~10~девушек.
Сколько существует способов составить компанию, в~которой было~бы одинаковое
число юношей и~девушек?

\item
Банда из~25 гангстеров устроила перестрелку, в~которой каждый из~них получил
пулю от~ближайшего гангстера
(они сидели в~укрытиях и~не~перемещались, все попарные расстояния между ними
были различны).
Докажите, что какой-то гангстер не~подстрелил никого.

\item
Докажите, что у~выпуклого многогранника найдутся две грани с~одинаковым числом
сторон.

\item
Дан выпуклый $n$-угольник такой, что никакие три его диагонали не~пересекаются
в~одной точке.
Найдите количество точек пересечения диагоналей данного многоугольника
(не~являющиеся вершинами многоугольника).

\end{problems}


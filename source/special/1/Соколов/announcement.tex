% $date: 2--5 июня 2018 г.

\worksheet*{Триангуляция Делоне}

% $authors:
% - Артемий Алексеевич Соколов

% $style[-announcement]:
% - .[announcement]

\emph{Триангуляцией} назовем разбиение плоскости на~треугольники таким образом,
чтобы любые два треугольника либо не~пересекались, либо имели общую сторону или
вершину.
Понятно, что триангуляций с~заданным множеством вершин существует очень много.
Мы познакомимся в~некотором смысле с~самой <<хорошей>>:
триангуляцией Делоне, в~которой описанная окружность каждого треугольника
не~содержит внутри других точек множества.
Также мы узнаем много других свойств этого разбиения, и~обсудим, как можно его
строить.


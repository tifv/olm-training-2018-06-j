% $date: 2--5 июня 2018 г.

\worksheet*{Семейства множеств}

% $authors:
% - Дмитрий Андреевич Захаров

% $style[-announcement]:
% - .[announcement]

В~курсе мы будем изучать различные вопросы такого вида: предположим, у~нас есть
набор конечных множеств, которые удовлетворяют каким-то дополнительным
ограничениям (например, множества попарно не~вложены или попарно пересекаются).
Можем~ли мы тогда сказать что-то про структуру и~количество этих множеств?
Я расскажу про несколько подходов к~решению таких задач.
Первые 2 лекции будут посвящены комбинаторным методам, главной теоремой для нас
будет знаменитая теорема Эрдеша-Ко-Радо о~максимальном размере пересекающегося
семейства.
Потом мы увидим, как теория чисел и~алгебра приходят на~помощь, когда
комбинаторика перестает работать.

Для понимания курса специальных знаний не~нужно.
Желательно иметь опыт работы <<по~модулю $p$>>.
Для зачета нужно будет сдать несколько задач из~листка.
 

% $date: 2--5 июня 2018 г.

\worksheet*{Группы подстановок}

% $authors:
% - Александр Савельевич Штерн

% $style[-announcement]:
% - .[announcement]

\vspace{-1ex}
\emph{Программа курса}
\vspace{+1ex}

\textbf{Занятие\;1.}\enspace
Повторение без доказательства: движения плоскости, теорема Шаля.
\\
Самосовмещения плоских фигур.
Задание самосовмещения подстановкой на~множестве вершин.
Определение подстановки.
Композиция подстановок.
Типы подстановок (транспозиции и~циклы).

\textbf{Занятие\;2.}\enspace
Четные и~нечетные подстановки.
Минимальные порождающие множества симметрической группы.
\\
Некоторые движения трехмерного пространства.
Самосовмещения правильного тетраэдра и куба. 

\textbf{Занятие\;3.}\enspace
Конечные подгруппы группы движений.
Централизатор подстановки.
Орбита и стабилизатор элемента.
Теорема об~индексе стабилизатора.

\textbf{Занятие\;4.}\enspace
Лемма Бернсайда о~среднем числе неподвижных точек и~ее приложения
в~комбинаторике.


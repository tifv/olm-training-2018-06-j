% $date: 2018-06-08
% $timetable:
%   g9r1:
%     2018-06-05:
%       2:
%   g9r2:
%     2018-06-07:
%       2:

\worksheet*{Максимум на~конце отрезка}

% $authors:
% - Леонид Андреевич Попов

%\begingroup
%    \def\abs#1{\lvert #1 \rvert}%

\begin{problems}

%\item
%Неотрицательные числа $a$, $b$, $c$, $A$, $B$, $C$, $k$ таковы, что
%$a + A = b + B = c + C = k$.
%Докажите, что
%\[
%    a B + b C + c A \leq k^2
%\, . \]

\itemy{0}
Пусть $a$~--- положительное число.
Найдите максимальное возможное значение суммы
\[
    \sum_{k = 1}^{n}
        (a - a_{1}) (a - a_{2})
        \ldots
        (a - a_{k-1}) a_{k} (a - a_{k+1})
        \ldots
        (a - a_{n})
\, , \]
где числа $a_{1}, a_{2}, \ldots, a_{n}$ принадлежат отрезку $[0; a]$.

\item
Пусть $0 \leq a, b, c, d \leq 1$.
Докажите, что
\[
    (1 - a) (1 - b) (1 - c) (1 - d) + a + b + c + d
\geq
    1
\, . \]

\item
Докажите, что площадь треугольника, лежащего внутри единичного квадрата,
не~превосходит $1/2$.

\item
Пусть $n \geq 2$ и~$0 \leq x_{i} \leq 1$ для любого $i = 1, 2, \ldots, n$.
Докажите, что
\[
    (x_{1} + x_{2} + \ldots + x_{n})
    - (x_{1} x_{2} + x_{2} x_{3} + \ldots + x_{n} x_{1})
\leq
    \left[ \frac{n}{2} \right]
\, . \]

%\item
%Найдите максимальное значение суммы
%\[
%    S_{n}
%=
%    a_{1} (1 - a_{2}) + a_{2} (1 - a_{3}) + \ldots + a_{n} (1 - a_{1})
%\, , \]
%где $\frac{1}{2} \leq a_{i} \leq 1$ для любого $i = 1, 2, \ldots, n$.

\item
Пусть действительные числа $x$, $y$, $z$ лежат на~отрезке $[0; 1]$.
Докажите, что
\[
    3 (x^2 y^2 + y^2 z^2 + x^2 z^2) - 2 x y z (x + y + z)
\leq
    3
\, . \]

%\item
%Пусть $n$~--- натуральное число и $x_{i} \in [0; 1]$, $i = 1, 2, \ldots, n$.
%Найдите максимум суммы
%\[
%    \sum_{1 \leq i < j \leq n}
%        \abs{x_{i} - x_{j}}
%\, . \]

\item
Докажите, что для любых чисел $a$, $b$, $c$ из~отрезка $[0; 1]$ выполнено
неравенство:
\[
    \frac{a}{b + c + 1} + \frac{b}{c + a + 1} + \frac{c}{a + b + 1} +
    (1 - a) (1 - b) (1 - c)
\leq
    1
\, . \]

\item
Докажите, что если $1 \leq x_{k} \leq 2$, $k = 1, 2, \ldots, n$, то%
\[
    \left(
        \sum_{k=1}^{n}
            x_{k}
    \right)
    \cdot
    \left(
        \sum_{k=1}^{n}
            \frac{1}{x_{k}}
    \right)^2
\leq
    n^3
\, . \]

%\item
%Пусть $A_{1} A_{2} \ldots A_{n}$~--- выпуклый многоугольник.
%Для каждой стороны $A_{i} A_{i+1}$ через $S_{i}$ обозначим наибольшую площадь
%треугольника, одной стороной которого является отрезок $A_{i} A_{i+1}$,
%а~третьей вершиной~--- вершина данного многоугольника
%(при $i = n$ рассматривается сторона $A_{n} A_{1}$).
%Пусть площадь многоугольника равна $S$.
%Докажите, что $\sum_{i=1}^{n} S_{i} \geq 2 S$.

%\item
%Пусть $n$~--- натуральное число,
%$x_{1}$, \ldots, $x_{n}$~--- действительные числа.
%Докажите неравенство
%\[
%    \sum_{i=1}^{n}
%        \sum_{j=1}^{n}
%            \abs{x_{i} + x_{j}}
%\geq
%    n
%    \sum_{i=1}^{n}
%        \abs{x_{i}}
%\, . \]

\item
Дан связный граф $G$, в~вершинах которого расставили неотрицательные числа
с~суммой 1, после чего для каждого ребра посчитали произведение чисел в~его
концах и~сложили полученные числа для всех ребер.
Выразите максимальное возможное значение рассмотренной суммы через размер
максимальной клики в~графе.
\par
\emph{Кликой} неориентированного графа называется подмножество его вершин,
любые две из~которых соединены ребром.

%\item
%В~единичный куб вписан выпуклый многогранник так, что его проекции на~грани
%куба полностью их покрывают.
%Найдите минимум объема этого многогранника.

\end{problems}

%\endgroup % \def\abs


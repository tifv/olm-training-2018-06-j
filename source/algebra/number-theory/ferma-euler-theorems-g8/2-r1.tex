% $date: 2018-06-11
% $timetable:
%   g8r1:
%     2018-06-11:
%       2:

\worksheet*{Ферма и~Эйлер~--- 2}

% $authors:
% - Андрей Борисович Меньщиков

\begingroup
    \ifdefined\threedotcolon
        \def\kratno{\mathrel{\threedotcolon}}%
    \fi

\begin{problems}

\item
Найдите остаток при делении $4^{1008}$ на\\
\subproblem 2017;
\quad
\subproblem 105;
\quad
\subproblem 2018.

%\item
%Докажите, что $2^{3^k} + 1$ делится на~$3^{k+1}$.

\item
Докажите, что $2^{n!} - 1 \kratno n$ для любого нечетного натурального $n$.

\item
Докажите, что число $99^{100} + 100^{99}$~--- составное.

\item
Пусть $p > 5$~--- простое число.
Докажите, что
\(
    \underbrace{111{\ldots}11}_{\text{$p-1$}}
\kratno
    p
\).
%\subproblem
%\(
%    \underbrace{11{\ldots}1}_{p}
%    \underbrace{22{\ldots}2}_{p}
%    \underbrace{33{\ldots}3}_{p}
%    \ldots
%    \underbrace{99{\ldots}9}_{p}
%    - 123456789
%\kratno
%    p
%\).

\item
Дана последовательность $a_n = 1^n + 2^n + 3^n + 4^n + 5^n$.
Найдутся~ли в~ней пять последовательных членов, делящихся на~$98765$?

\item
При каких простых~$p$ число $5^{p^2} + 1$ делится на~$p$?

%\item
%Докажите, что для составного числа $561$ справедлив аналог малой теоремы Ферма:
%если $(a, 561) = 1$, то~$a^{560} \equiv 1 \pmod {561}$.

%\item
%Докажите, что если $p$~--- простое число, отличное от~$2$ и~$5$, то~длина
%периода разложения $1 / p$ в~десятичную дробь делит $p - 1$.

%\item
%При каких натуральных $n$ число $n^2 + n + 1$ делится на~$101$?

\item
На~какие три цифры оканчивается число $7^{9999}$?

\item
Какой остаток дает число $42^{42^{42}}$ при делении на~$2017$?

\item
\subproblem
Докажите, что $n^{84} - n^{4} \kratno 20400$ для любого натурального $n$.
\\
\subproblem
Можно~ли $20400$ заменить на~какое-нибудь большее число, чтобы утверждение
осталось верным?

\item
Пусть $p$ и~$q$~--- различные простые числа.
Какой остаток дает число $p^{q} + q^{p}$ при делении на~$p q$?

\item
Конечно~ли множество чисел вида вида $2^{n} - 1$, у~которых более миллиона
различных простых делителей?

% В разнобой
%\item
%Пусть $p$ и~$p + 2$~--- простые числа.
%Докажите, что $2 p (p + 1) (p + 2)$ является общим делителем чисел
%$p^{p+2} - p$ и~$(p + 2)^{p} - p - 2$.

% для 7-1
\item
Натуральные числа $a$, $b$, $c$ таковы, что
\( a^{504} + b^{504} + c^{504} \kratno 2018 \).
Докажите, что и~$a b c \kratno 2018$.

% В разнобой
%\item
%Докажите, что для любого простого числа~$p$ найдется число вида $2018^{n} - n$,
%делящееся на~$p$.

% для 7-1
\item
При каких натуральных~$n$ для каждого натурального $k \geq n$ существует
натуральное число с~суммой цифр $k$, делящееся на~$n$?
% http://math.mosolymp.ru/upload/files/2017/khamovniki/7/170311_TCHshnyie_teoremyi_razbor_DZ.pdf

%\item
%Докажите, что для любого натурального~$n$ существует число с~суммой цифр $n$,
%делящееся на~$n$.

% Вдогонку к~Вильсону
% http://math.mosolymp.ru/upload/files/2017/khamovniki/7/170401_TCH_dobavka.pdf

\end{problems}

\endgroup % \def\kratno


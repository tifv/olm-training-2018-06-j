% $date: 2018-06-09
% $timetable:
%   g8r1:
%     2018-06-09:
%       2:

\worksheet*{Ферма и~Эйлер}

% $authors:
% - Андрей Борисович Меньщиков

\begingroup
    \ifdefined\mathup%
        \providecommand\eulerphi{\mathup{\mupphi}}%
    \fi%
    \ifdefined\upphi%
        \providecommand\eulerphi{\upphi}%
    \fi
    \providecommand\eulerphi{\phi}%
    \def\binom#1#2{\mathrm{C}_{#1}^{#2}}%

\claim{Определение 1}
\emph{Полная система вычетов} по~модулю $m$~--- это все остатки от~деления
на~$m$, встречающиеся по~одному разу.

\claim{Определение 2}
\emph{Приведённая система вычетов} по~модулю $m$~--- это все остатки от~деления
на~$m$, взаимно простые с~$m$, встречающиеся по~одному разу.

\claim{Малая теорема Ферма (1)}
Пусть $p$~--- простое число, а~$a$ не~делится на~$p$.
Тогда $a^{p-1} \equiv 1 \pmod{p}$.

\claim{Малая теорема Ферма (2)}
Пусть $a$~--- целое число, а~$p$~--- простое.
Тогда $a^{p} \equiv a \pmod{p}$.

\begin{problems}

\item
\subproblem
Вспомните бином Ньютона
\(
    (a + b)^{n}
=
    a^{n} + \ldots + \binom{n}{k} a^{n-k} b^{k} + \ldots + b^{n}
\),
а~также докажите, что $\binom{p}{l}$ делится на~$p$, где $p$~--- простое число,
а~$l < p$~--- натуральное.
\\
\subproblem
Докажите малую теорему Ферма с~помощью индукции.

\item%
\label{/algebra/number-theory/ferma-euler-theorem-g8/1-r1/:problem:proof-2}%
\subproblem
Пусть $a$ не~делится на~$p$.
Докажите, что множество остатков чисел
$1 \cdot a$, $2\cdot a$, \ldots, $(p - 1) \cdot a$ при делении на~$p$ совпадает
с~множеством $\{ 1, 2, \ldots, p - 1 \}$.
\\
\subproblem
Выведите отсюда малую теорему Ферма.

\item
Сколько существует способов раскрасить вершины правильного $p$-угольника
в~$a$ цветов?
(Раскраски, совмещающиеся поворотом, считаются одинаковыми.)
Найдите формулу и~выведите из~нее малую теорему Ферма.

\item
\label{/algebra/number-theory/ferma-euler-theorem-g8/1-r1/:problem:proof-4}%
Вершинам правильного $(p - 1)$-угольника присвоим различные ненулевые остатки
при делении на~$p$.
Проведем из~остатка~$k$ стрелку в~остаток~$k a$.
\\
\subproblem
Докажите, что из~каждой точки выходит одна стрелка, и~в~каждую точку входит
одна стрелка, т.\,е. все стрелки разбиваются на~циклы.
\\
\subproblem
Докажите, что у~всех этих циклов одна и~та~же длина, делящая $p - 1$,
и~выведите отсюда малую теорему Ферма.

\item
Верна~ли малая теорема Ферма в~обратную сторону?
Если для натурального $k > 1$ и~всех целых $a$ верно $a^{k} \equiv a \pmod k$,
то~обязательно~ли $k$~--- простое число?

\end{problems}

\claim{Определение}
Пусть $n$~--- натуральное число.
\emph{Функция Эйлера} $\eulerphi(n)$ определяется как количество чисел,
не~превосходящих $n$, взаимно простых с~$n$.

\claim{Упражнение}
Докажите, что если $n > 2$, то~$\eulerphi(n)$ четно.

\claim{Теорема Эйлера}
Натуральные $a$ и~$n$ взаимно просты.
Тогда $a^{\eulerphi(n)} \equiv 1 \pmod{n}$.

\begin{problems}

\item
\subproblem
Приведите рассуждения, аналогичные задаче~%
\ref{/algebra/number-theory/ferma-euler-theorem-g8/1-r1/:problem:proof-2},
и~докажите теорему Эйлера.
\\
\subproblem
Приведите рассуждения, аналогичные задаче~%
\ref{/algebra/number-theory/ferma-euler-theorem-g8/1-r1/:problem:proof-4},
и~докажите теорему Эйлера.

\item
\subproblem
Пусть $p$ и~$q$~--- различные простые числа.
Найдите $\eulerphi(p^\alpha)$ и~$\eulerphi(p q)$.

\subproblem
Докажите, что если $a$ и~$b$ взаимно просты,
то~\(
    \eulerphi(a b) = \eulerphi(a) \cdot \eulerphi(b)
\).

\subproblem
Пусть \(
    n
=
    p_{1}^{\alpha_{1}} p_{2}^{\alpha_{2}} \ldots p_{k}^{\alpha_{k}}
\)~---
разложение $n$ на~простые множители.
Докажите, что
\[
    \eulerphi(n)
=
    n
    \left( 1 - \frac{1}{p_{1}} \right)
    \left( 1 - \frac{1}{p_{2}} \right)
    \ldots
    \left( 1 - \frac{1}{p_{k}} \right)
. \, \]

\item\claim{Усиление теоремы Эйлера}
Пусть \(
    m
=
    p_{1}^{\alpha_{1}} p_{2}^{\alpha_{2}} \ldots p_{k}^{\alpha_{k}}
\)~---
разложение $m$ на~простые множители,
\(
    s = \operatorname{\text{\rm НОК}} \bigl(
        \eulerphi(p_{1}^{\alpha_{1}}),
        \eulerphi(p_{2}^{\alpha_{2}}), \ldots,
        \eulerphi(p_{k}^{\alpha_{k}})
    \bigr)
\).
Докажите, что для любого целого $a$, взаимно простого с~$m$, верно
$a^{s} \equiv 1 \pmod{m}$.

\end{problems}


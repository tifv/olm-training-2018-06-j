% $date: 2018-06-12
% $timetable:
%   g9r1:
%     2018-06-12:
%       2:

\worksheet*{Разнобой по~теории чисел}

% $authors:
% - Леонид Андреевич Попов

\begingroup
    \def\abs#1{\lvert #1 \rvert}%
    \def\divides{\mathrel{\vert}}%

\begin{problems}

\item
Пусть $a$~--- нечетное число.
Докажите, что числа $a^{2^{n}} + 2^{2^{n}}$ и~$a^{2^{m}} + 2^{2^{m}}$ взаимно просты
при любых натуральных $n \neq m$.

\item
Пусть $a$ и~$b$~--- такие различные натуральные числа, что $a b (a + b)$
делится на~$a^2 + a b + b^2$.
Докажите, что $\abs{a - b} > \sqrt[3]{ab}$.
% Андрееску, стр. 15

\item
Найдите все натуральные $n$, для которых $n^{n} + 1$ и~$(2n)^{2n} + 1$ являются
простыми.
% http://artofproblemsolving.com/community/c6h381151p2110492

\item
Найдите все натуральные $n$ такие, что
\[
    n \divides 1^{n} + 2^{n} + \ldots + (n - 1)^{n}
\, . \]
% http://artofproblemsolving.com/community/c6t45788f6h1110315

\item
Докажите, что в~последовательности чисел $2^{n} - 3$, где $n = 1, 2, \ldots$,
существует набор из~$2018$ попарно взаимно простых чисел.
% 7.1.9 Number Theory: Structures, Examples, and Problems by Titu Andreescu

\end{problems}

\endgroup % \def\abs \def\divides


% $date: 2018-06-14
% $timetable:
%   g8r1:
%     2018-06-14:
%       1:
%   g8r2:
%     2018-06-14:
%       2:

\worksheet*{Числовой разнобой}

% $authors:
% - Андрей Борисович Меньщиков

\begin{problems}

\item
Петя посчитал НОК всех чисел от~1 до~1000, а~Вася -- всех чисел от~501 до~1000.
У~кого результат получился больше и~во~сколько раз?

\item
Пусть $p$~--- простое число.
Докажите, что $(2 p - 1)! - p \kratno p^2$.

\item
Целые числа $m$ и~$n$ взаимно просты.
Какое наибольшее значение может принимать $(m + 2018 n, n + 2018 m)$?

\item
Докажите, что для любого натурального~$k$ существует сколь угодно длинный
отрезок из~натуральных чисел, у~каждого из~которых есть хотя~бы $k$ различных
простых делителей.

\item
Пусть $p$ и~$p+2$~--- простые числа.
Докажите, что $2 p (p + 1) (p + 2)$ является общим делителем чисел
$p^{p+2} - p$ и~$(p + 2)^{p} - p - 2$.

%\item
%По~бесконечной шахматной доске ходит $(m, n)$-крокодил, который может за~один
%ход сдвинуться на~$m$ клеток по~горизонтали или вертикали, а~затем~---
%на~$n$ клеток в~перпендикулярном направлении.
%При каких $m$ и~$n$ $(m,n)$--крокодил сможет попасть из~любой клетки доски
%в~любую другую?

\item
Докажите, что для любого простого числа~$p$ найдётся число вида $2018^{n} - n$,
делящееся на~$p$.

%\item
%Докажите, что для любого натурального~$n$ существует число с~суммой цифр $n$,
%делящееся на~$n$.

\item
Для каких натуральных~$n$ число $(n - 1)! + 1$ является точной степенью числа~$n$?

\end{problems}


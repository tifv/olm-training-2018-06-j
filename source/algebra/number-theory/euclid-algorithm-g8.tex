% $date: 2018-06-01
% $timetable:
%   g8r1:
%     2018-06-01:
%       1:
%   g8r2:
%     2018-06-01:
%       2:

% $build$matter[print]: [[.], [.]]

\worksheet*{НОД. Алгоритм Евклида}

% $authors:
% - Антон Сергеевич Гусев

\begin{problems}
    \def\gcd{\operatorname{\text{\rmfamily НОД}}}%

\item
Докажите, что дробь $\frac{n^2 + 7 n + 13}{n + 3}$ несократима при всех
натуральных $n$.

\item
Найдите $\gcd(11! - 20, 10! - 20)$.

\item
Найдите $\gcd(11{\ldots}1, 11{\ldots}1)$:
\\
\subproblem в~первом числе 100 единиц, во~втором 60;
\\
\subproblem в~первом числе $m$~единиц, во~втором $n$.

\item
Найдите $\gcd(2^{100} - 1, 2^{60} - 1)$.

\item
Докажите, что $\gcd(a^{m} - 1, a^{n} - 1) = a^{\gcd(m,n)} - 1$.

\item
Найдите НОД
\\
\subproblem
всех шестизначных чисел, составленных из~цифр 1, 2, 3, 4, 5, 6 без повторений;
\\
\subproblem
всех чисел, состоящих из~5 единиц и~7 двоек.

\item
На~столе лежат две кучки: в~одной 1573 ореха, в~другой 97900.
Двое играют в~игру:
очередным ходом игрок может разложить любую кучку на~две, одна из~которых
совпадает по~размерам с~уже имеющейся на~столе.
Кто выиграет при правильной игре?

\item
Есть шоколадка в~форме равностороннего треугольника со~стороной~$n$,
разделенная на~равносторонние треугольники со~стороной~1.
Двое играют в~игру.
За~ход можно отломить треугольник с~целой стороной, съесть его, а~остаток
передать сопернику.
Также можно съесть последний треугольник со~стороной~1.
Проигрывает тот, кто не~может сделать ход.
Кто выиграет при правильной игре?

\end{problems}


% $date: 2018-06-05
% $timetable:
%   g8r1:
%     2018-06-05:
%       2:
%   g8r2:
%     2018-06-05:
%       1:

\worksheet*{Сравнения}

% $authors:
% - Андрей Борисович Меньщиков

\begingroup
    \ifdefined\threedotcolon
        \def\kratno{\mathrel{\threedotcolon}}%
    \fi

\begin{problems}

% Для обсуждения
%\item
%Натуральные числа $a$ и~$b$ таковы, что $3 a + 7 b$ дает остаток $5$ при
%делении на~$11$.
%А~какой остаток дает $2 a + b$ при делении на~$11$?

\item
Делится~ли $5^{70} + 6^{70}$ на~$61$?

\item
Целые числа $a$ и~$b$ таковы, что $N = (31 a + 57 b) (57 a + 31 b)$
делится на~$11$.
Докажите, что $N$ делится на~$121$.

\item
У~числа $2018^{2018}$ нашли сумму цифр.
У~результата опять нашли сумму цифр, и~т.\,д., пока не~получилась цифра.
А~что это за~цифра?

\item
Докажите, что произведение первых $n$ простых чисел, увеличенное на~$1$,
не~является точным квадратом.

%\item
%Пусть $p$~--- простое число, $k$~--- натуральное, не~делящееся на~$p$.
%Докажите, что среди остатков чисел
%$0 \cdot k$, $1 \cdot k$, $2 \cdot k$, $\ldots$, $(p - 1) \cdot k$ при делении
%на~$p$ все возможные остатки встречаются ровно по~одному разу.

%\item
%У~числа $n^2 + 2 n$ последняя цифра равна $4$.
%А~какая у~него предпоследняя цифра?

%\item
%Докажите, что разность произведения всех нечетных чисел от~$1$ до~$2015$
%и~произведения всех четных чисел от~$2$ до~$2016$ делится на~$2017$.

\item
Существует~ли степень двойки, из~которой перестановкой цифр можно получить
другую степень двойки?
(Цифру 0 ставить на~первое место нельзя.)

\item
Пусть $p$~--- простое число.
Может~ли сумма чисел $2^{p}$ и~$3^{p}$ оказаться степенью (выше первой)
натурального числа?

\item
Пусть $p$ и~$q$~--- последовательные нечетные числа.
Докажите, что $p^{p} + q^{q} \kratno p + q$.

\item
Можно~ли разбить числа $1, 2, 3, \ldots, 100$ на~три группы так, чтобы в~первой
группе сумма чисел делилась на~$102$, во~второй~--- на~$203$, а~в~третьей~---
на~$304$?

%\item
%Пусть $n$~--- натуральное.
%Может~ли число $2^{n} + 15$ быть точным кубом?

\item
Для натуральных $a$ и~$n$ докажите, что
$a^{2n+3} + (a - 1)^{n} \kratno a^2 - a + 1$.

%\item
%При каких натуральных $n$ число $n^4 + n^3 + n^2 + n + 1$ делится на~$2017$?

\item
На~доске записано число $2018$.
За~одну операцию разрешается в~любое место числа вставить две одинаковые цифры.
Можно~ли в~результате нескольких таких операций получить число, кратное $583$?

\item
При каких целых $k$ число $a^3 + b^3 + c^3 - k a b c$ делится на~$a + b + c$
при любых целых $a$, $b$, $c$ с~ненулевой суммой?

\item
Последовательность задана следующим образом:
$a_{1} = a_{2} = 1$, $a_{k+2} = a_{k} \cdot a_{k+1} + 1$ при всех
натуральных~$k$.
Докажите, что при всех натуральных $n \geq 7$ число $a_{n} - 3$ является
составным.

\end{problems}

\endgroup % \def\kratno


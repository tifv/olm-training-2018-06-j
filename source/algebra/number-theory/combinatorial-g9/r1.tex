% $date: 2018-06-09
% $timetable:
%   g9r1:
%     2018-06-09:
%       2:
%   g9r2:
%     2018-06-12:
%       1:

\worksheet*{Комбинаторная теория чисел}

% $authors:
% - Леонид Андреевич Попов

\begin{problems}

\item
На~доске написано 10 натуральных чисел.
Докажите, что из~этих чисел можно выбрать несколько чисел и~расставить между
ними знаки <<+>> и~<<-->> так, чтобы полученная в~результате алгебраическая
сумма делилась на~1001.

\item
Дана бесконечная вправо последовательность цифр и~натуральное число~$l$.
Докажите, что можно выбрать несколько цифр подряд, образующих число, делящееся
на~$l$, если $l$~--- нечетное число, не~делящееся на~$5$.

%\item
%Докажите, что одно из~сравнений
%\[
%    x^2 \equiv 1 \pmod{101}
%\, , \quad
%    x^2 \equiv 2 \pmod{101}
%\, , \quad \ldots \, , \quad
%    x^2 \equiv 50 \pmod{101}
%\]
%не~имеет решений.

\item
Докажите, что если $p$~--- простое число, то~разрешимо сравнение
\[
    1 + x^2 + y^2 \equiv 0 \pmod{p}
\, . \]

\item
Докажите, что найдется число, представимое в~виде суммы четырех квадратов целых
чисел более, чем миллионом способов.

\item
Дана строчка из~$25$ цифр.
Всегда~ли можно расставить в~этой строчке знаки арифметических операций
$+$, $-$, $\times$, $:$ и~скобки так, чтобы образовалось числовое выражение,
равное~$0$?
Последовательно стоящие цифры можно объединять в~числа, но~порядок цифр
изменять нельзя.

\item
Назовем число \emph{хорошим,} если его простые делители принадлежат множеству
$\{ 2, 3, 5 \}$.
Записали $81$ хорошее число.
Докажите, что из~них можно выбрать четыре числа, произведение которых точная
четвертая степень.

\item
Дано натуральное число~$k$.
На~бесконечной клетчатой плоскости каждая клетка является суверенным
государством, а~на~каждом ребре стоит таможня, взимающая натуральное число
талеров в~качестве взятки за~ее пересечение (в~обоих направлениях~---
одинаковое, но, возможно, различное для разных границ).
Докажите, что существует такой замкнутый маршрут, не~заходящий ни~в~какую
клетку дважды, что суммарная взятка на~нем кратна $k$.

\end{problems}


% $date: 2018-06-12
% $timetable:
%   g8r1:
%     2018-06-12:
%       2:
%   g8r2:
%     2018-06-12:
%       1:

\worksheet*{КТО и~Вильсон}

% $authors:
% - Андрей Борисович Меньщиков

\begingroup
    \ifdefined\mathup
        \providecommand\eulerphi{\mathup{\mupphi}}%
    \fi
    \ifdefined\upphi
        \providecommand\eulerphi{\upphi}%
    \fi
    \providecommand\eulerphi{\phi}%

\begin{exercises}

\item
Найдите наименьшее натуральное число, дающее при делении на~2, 3, 5, 7, 11
остатки 1, 2, 4, 6, 10 соответственно.

\item\jeolmlabel{/algebra/number-theory/crt-wilson-theorem-g8/:exercise:2}
\subproblem
Решите в~целых числах уравнение $10 x + 8 = 11 y + 10$.
\\
\subproblem
Найдите все числа, дающие остаток 2 при делении на~8 и~остаток 11 при делении на~13.

\end{exercises}

\claim{Китайская теорема об~остатках}
Пусть целые числа $m_{1}$, \ldots, $m_{n}$ попарно взаимно просты,
$m = m_{1} \ldots m_{n}$, а~$a_{1}$, \ldots, $a_{n}$~--- произвольные целые
числа.
Тогда существует единственное целое число $x$ такое, что $0 \leq x < m$ и%
\[ \left\{ \begin{aligned} &
    x \equiv a_1 \pmod {m_1}
, \\ &
    x \equiv a_2 \pmod {m_2}
, \\ & \ldots \\ &
    x \equiv a_n \pmod {m_n}
. \end{aligned} \right. \]

\begin{problems}

\item
\subproblem
Натуральные числа $m_{1}$, \ldots, $m_{n}$ попарно взаимно просты.
Докажите, что число $x = (m_{2} m_{3} \ldots m_{n})^{\eulerphi(m_{1})}$
является решением системы
\[ \left\{ \begin{aligned} &
    x \equiv 1 \pmod {m_1}
, \\ &
    x \equiv 0 \pmod {m_2}
, \\ & \ldots \\ &
    x \equiv 0 \pmod {m_n}
. \end{aligned} \right. \]
\par
\subproblem
Найдите в~явном виде число~$x$, удовлетворяющее КТО.

\item
Натуральные числа $m_{1}$, $m_{2}$, \ldots, $m_{n}$ попарно взаимно просты.
Докажите, что если числа $x_{1}$, $x_{2}$, \ldots, $x_{n}$ пробегают полные
системы вычетов по~модулям $m_{1}$, $m_{2}$, \ldots, $m_{n}$ соответственно,
то~число
\(
    x
=
    x_{1} m_{2} \ldots m_{n} + m_{1} x_{2} \ldots m_{n}
    + \ldots +
    m_{1} m_{2} \ldots m_{n-1} x_{n}
\) пробегает полную систему вычетов по~модулю $m_{1} m_{2} \ldots m_{n}$.
Выведите отсюда КТО.

\item
Проникнитесь упражнением~%
\jeolmref{/algebra/number-theory/crt-wilson-theorem-g8/:exercise:2},
и~докажите КТО по~индукции.

\end{problems}

\claim{Теорема Вильсона}
Пусть $p$~--- простое число.
Тогда $(p - 1)! \equiv -1 \pmod p$.

\begin{problems}

\item
Докажите, что для любого целого $a$, не~делящегося на~$p$, найдется целое $b$
такое, что $a b \equiv 1 \pmod p$.
Попробуйте разбить все ненулевые остатки при делении на~$p$ на~пары обратных,
и~докажите теорему Вильсона.

\item
Сколько существует ориентированных замкнутых ломаных, проходящих через все
вершины правильного $p$-угольника?
(Ломаные, совмещающиеся поворотом, считаются одинаковыми.)
Найдите формулу и~выведите из~нее теорему Вильсона.

\item
Верна~ли теорема Вильсона в~обратную сторону?
Если для натурального $p > 1$ верно $(p - 1)! \equiv -1 \pmod p$,
то~обязательно~ли $p$~--- простое число?

%\item
%Генерал построил солдат в~колонну по~4, но~при этом солдат Иванов остался
%лишним.
%Тогда генерал построил солдат в~колонну по~5.
%И~снова Иванов остался лишним.
%Когда~же и~в~колонне по~6 Иванов оказался лишним, генерал посулил ему наряд
%вне очереди, после чего в~колонне по~7 Иванов нашел себе место и~никого
%лишнего не~осталось.
%Сколько солдат могло быть у~генерала?

%В~разнобой
%\item
%Докажите, что для любого натурального~$k$ существует сколь угодно длинный
%отрезок из~натуральных чисел, каждое из~которых делится хотя~бы
%на~$k$ различных простых чисел.

%В~разнобой
%\item
%Пусть $p$~--- простое число.
%Докажите, что $(2 p - 1)! - p$ делится на~$p^2$.

\item
\subproblem
Докажите, что для любого простого числа~$p$ вида $4 k + 1$ существует целое
число~$n$ такое, что $n^2 \equiv -1 \pmod p$.
\\
\subproblem
Докажите, что для простых чисел~$p$ вида $4 k + 3$ таких $n$ не~существует.

%\item
%Для каких натуральных~$n$ число $(n - 1)! + 1$ является точной степенью числа $n$?

\end{problems}

\endgroup % \def\eulerphi


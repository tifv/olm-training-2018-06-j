% $date: 2018-06-02
% $timetable:
%   g8r1:
%     2018-06-02:
%       2:
%   g8r2:
%     2018-06-02:
%       1:

\worksheet*{Диофантовы уравнения}

% $authors:
% - Антон Сергеевич Гусев

\begin{problems}

\item
Пусть $(a, b) = 1$ и~$(x_0, y_0)$~--- некоторое целочисленное решения уравнения
$a x + b y = 1$.
Докажите, что все решения этого уравнения в~целых числах получаются по~формулам
$x = x_0 + k b$, $y = y_0 - k a$, где $k$~--- произвольное целое число.

\item
Как описать в~целых числах все решения уравнения $a x + b y = c$ при
произвольных целых $a$, $b$, $c$?

\item
Как при помощи алгоритма Евклида найти какое-нибудь решение уравнения
$a x + b y = c$ (при условии, что оно существует)?

\item
Решите в~целых числах уравнения:
\\
\subproblem $5 x - 4 y = 1$;
\qquad
\subproblem $9 x + 15 y = 4$;
\qquad
\subproblem $1500 x + 501 y = 2001$;
\\
\subproblem $-27 x + 12 y = 15$;
\qquad
\subproblem $14 x - 68 y = 8$.

\item
Докажите, что любое натуральное число можно представить в~виде отношения 7-й
степени какого-то числа и~5-й степени какого-то числа.

\item
На~складе имеется тушенка в~банках по~350\,г и~по~425\,г.
Для проведения операции <<Буря в~стакане>> требуется 15\,кг тушенки.
Подскажите главному интенданту, сколько и~каких банок заказать на~складе.

\item
Школьники ходили купаться на~реку через большой песчаный пляж.
Шедший последним Степа аккуратно провел на~песке две черты, перпендикулярных
направлению движения ребят, на~расстоянии 10~метров друг от~друга, и~насчитал
между ними ровно 559 следов.
Сколько семиклассников ходило на~реку, если известно, что длина шага каждого
из~них составляет 55\,см?

\item
Остап Бендер организовал раздачу слонов населению.
На~раздачу явилось 11~членов профсоюза и~15 не-членов, причем Остап раздавал
слонов поровну всем членам профсоюза и~поровну не-членам
(всем хотя~бы по~одному слону!).
Оказалось, что существует лишь один способ такой раздачи
(так, чтобы раздать всех слонов).
Какое наибольшее число слонов могло быть у~O.\,Бендера?

\item
Все натуральные числа поделены на~хорошие и~плохие.
Известно, что если число~$A$ хорошее, то~и~число $A + 6$ тоже хорошее, а~если
число~$B$ плохое, то~и~число $B + 15$ тоже плохое.
Когда взяли $N$ первых чисел, оказалось, что среди них плохих чисел в~три раза
меньше, чем хороших.
Чему равно $N$?

\item
Докажите, что если $(a_1, a_2, \ldots, a_n) = 1$, то~уравнение
\[
    a_{1} x_{1} + a_{2} x_{2} + \ldots + a_{n} x_{n} = 1
\]
разрешимо в~целых числах.

\item
Отметим на~прямой красным цветом все точки вида $81 x + 100y$, где $x$, $y$~---
натуральные, и~синим цветом~--- все остальные точки.
Найдите на~прямой такую точку, что любые симметричные относительно нее целые
точки закрашены в~разные цвета.

\end{problems}


% $date: 2018-06-05
% $timetable:
%   g8r1:
%     2018-06-05:
%       1:
%   g8r2:
%     2018-06-05:
%       2:

% $build$matter[print]: [[.], [.]]

\worksheet*{Разобьем на~равнобедренные треугольники}

% $authors:
% - Александр Давидович Блинков

\begin{problems}

\item
Два угла треугольника равны $100^\circ$ и~$60^\circ$.
Покажите, как его разрезать на~два равнобедренных треугольника.

\item
Верно~ли, что любой треугольник можно разбить на~четыре равнобедренных
треугольника?

\item
Прямая разрезает равнобедренный треугольник на~два равнобедренных треугольника.
В~каком отношении эта прямая может разделить угол треугольника?

\item
Про треугольник, один из~углов которого равен $120^\circ$, известно, что его
можно разрезать на~два равнобедренных треугольника.
Чему могут быть равны два других угла исходного треугольника?

\item
Один из~углов треугольника равен $40^\circ$.
Известно, что его можно разбить отрезком на~два равнобедренных треугольника.
Найдите остальные углы данного треугольника.

\item
Треугольник $ABC$ можно разрезать на~два равнобедренных треугольника двумя
способами, проводя прямые через вершину~$C$.
Найдите углы треугольника $ABC$.

\item
Биссектриса~$BD$ треугольника $ABC$ отсекает от~него равнобедренный
треугольник $BCD$, а~биссектриса~$CE$ отсекает от~$ABC$ тупоугольный
равнобедренный треугольник $ACE$.
Найдите углы треугольника $ABC$.

\item
В~равнобедренном треугольнике $ABC$ каждый из~углов содержит нецелое число
градусов.
Известно, что через одну из~вершин этого треугольника можно провести
прямолинейный разрез, разбивающий его на~два равнобедренных треугольника.
Найдите углы треугольника $ABC$.

\item
Диагональ четырехугольника делит его на~два равнобедренных прямоугольных
треугольника.
Найдите углы этого четырехугольника.

\item
Две стороны четырехугольника равны 1 и~4.
Одна из~диагоналей имеет длину 2 и~делит его на~два равнобедренных
треугольника.
Найдите периметр четырехугольника.

\item
Каждая из~трех прямых делит четырехугольник на~два равнобедренных треугольника.
Найдите углы этого четырехугольника.

\end{problems}



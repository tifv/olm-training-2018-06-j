% $date: 2018-06-07
% $timetable:
%   g9r1:
%     2018-06-07:
%       2:

\worksheet*{Теорема Паскаля}

% $authors:
% - Артемий Алексеевич Соколов

\begin{claim}{Теорема Паскаля}
Даны шесть точек $A$, $B$, $C$, $D$, $E$, $F$ на~одной окружности.
Тогда пересечения прямых $AB$ и~$DE$, $BC$ и~$EF$, $CD$ и~$FA$ лежат на~одной
прямой.
\end{claim}

\begin{problems}

\item
В~окружность вписан шестиугольник $ABCDEF$.
Отрезок~$AC$ пересекается с~отрезком~$BF$ в~точке~$X$,
$BE$ с~$AD$~--- в~точке~$Y$ и~$CE$ с~$DF$ в~точке~$Z$.
Докажите, что точки $X$, $Y$ и~$Z$ лежат на~одной прямой.%
\footnote{Попробуйте найти на~картинке два подобных треугольника с~изогонально
сопряженными точками.}

%\item
%\emph{(Теорема Паскаля для треугольника)}
%Докажите, что в~неравнобедренном треугольнике точки пересечения касательных
%к~описанной окружности, восстановленных в~вершинах, с~противоположными
%сторонами лежат на~одной прямой.

\item
Окружность, проходящая через вершины $A$ и~$D$ основания трапеции $ABCD$,
пересекает боковые стороны $AB$, $CD$ в~точках $P$, $Q$, а~диагонали~---
в~точках $E$, $F$.
Докажите, что прямые $BC$, $PQ$, $EF$ пересекаются в~одной точке.

%\item
%Даны треугольник $ABC$ и~некоторая точка~$T$.
%Пусть $P$ и~$Q$~--- основания перпендикуляров, опущенных из~точки~$T$
%на~прямые $AB$ и~$AC$ соответственно, а~$R$ и~$S$~--- основания
%перпендикуляров, опущенных из~точки~$A$ на~прямые $TC$ и~$TB$ соответственно.
%Докажите, что точка пересечения прямых $PR$ и~$QS$ лежит на~прямой~$BC$.

\item
\emph{Теорема Паскаля для четырехугольника.}
Докажите, что прямая, соединяющая точки пересечения пар противоположных сторон
вписанного в~окружность четырехугольника, совпадает с~прямой, соединяющей точки
пересечения пар касательных к~этой окружности, восставленных в~противоположных
вершинах.

\item
Внутри треугольника $ABC$ отмечена точка~$P$.
Прямые $AP$, $BP$, $CP$ вторично пересекают описанную окружность треугольника
$ABC$ в~точках $A_{1}$, $B_{1}$, $C_{1}$ соответственно.
Докажите, что главные диагонали шестиугольника, полученного пересечением
треугольников $ABC$ и~$A_{1}B_{1}C_{1}$, пересекаются в~точке~$P$.

\item
Четырехугольник $ABCD$ вписан в~окружность с~центром~$O$.
Точка $X$ такова, что $\angle BAX = \angle CDX = 90^{\circ}$.
Докажите, что точка пересечения диагоналей четырехугольника $ABCD$ лежит
на~прямой~$XO$.

\item
Пусть $A'$~--- точка, диаметрально противоположная точке~$A$ в~описанной
окружности треугольника $ABC$ с~центром~$O$.
Касательная к~описанной окружности в~точке~$A'$ пересекает прямую~$BC$
в~точке~$X$.
Прямая~$OX$ пересекает стороны $AB$ и~$AC$ в~точках $M$ и~$N$.
Докажите, что $OM = ON$.

\item
На~сторонах $AB$ и~$AC$ остроугольного треугольника $ABC$ выбрали
соответственно точки $M$ и~$N$ так, что отрезок~$MN$ проходит через центр~$O$
описанной окружности треугольника $ABC$.
Пусть $P$ и~$Q$~--- середины отрезков $CM$ и~$BN$.
Докажите, что $\angle POQ = \angle BAC$.
% №3 с летних сборов 2014

\item
Треугольники $ABC$ и~$A'B'C'$ вписаны в~одну и~ту~же окружность, и~их
пересечением является шестиугольник.
Докажите, что главные диагонали шестиугольника пересекаются в~одной точке.

%\item
%На~стороне~$AB$ треугольника $ABC$ взята точка~$D$.
%В~угол $ADC$ вписана окружность, касающаяся изнутри описанной окружности
%треугольника $ACD$, а~в~угол $BDC$~--- окружность, касающаяся изнутри описанной
%окружности треугольника $BCD$.
%Оказалось, что эти окружности касаются отрезка~$CD$ в~одной и~той~же точке~$X$.
%Докажите, что перпендикуляр, опущенный из~$X$ на~$AB$, проходит через центр
%вписанной окружности треугольника $ABC$.

%\item
%В~треугольнике $ABC$ проведены высоты $AA_{1}$ и~$BB_{1}$
%и~биссектрисы $AA_{2}$ и~$BB_{2}$;
%вписанная окружность касается сторон $BC$ и~$AC$ в~точках $A_{3}$ и~$B_{3}$.
%Докажите, что прямые $A_{1}B_{1}$, $A_{2}B_{2}$, $A_{3}B_{3}$ пересекаются
%в~одной точке или параллельны.

%\item
%Равносторонний треугольник $ABC$ вписан в~окружность~$\Omega$ и~описан вокруг
%окружности~$\omega$.
%На~сторонах $AC$ и~$AB$ выбраны точки $P$ и~$Q$ соответственно так, что
%отрезок~$PQ$ проходит через центр треугольника $ABC$.
%Окружности $\Gamma_{b}$ и~$\Gamma_{c}$ построены на~отрезках $BP$ и~$CQ$ как
%на~диаметрах.
%Докажите, что окружности $\Gamma_{b}$ и~$\Gamma_{c}$ пересекаются в~двух
%точках, одна из~которых лежит на~$\Omega$, а~другая~--- на~$\omega$.

\end{problems}


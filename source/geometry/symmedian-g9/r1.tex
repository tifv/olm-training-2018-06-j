% $date: 2018-06-04
% $timetable:
%   g9r1:
%     2018-06-04:
%       2:
%   g9r2:
%     2018-06-07:
%       1:

\worksheet*{Симедиана и~гармонический четырехугольник}

% $authors:
% - Артемий Алексеевич Соколов

\begin{claim}{Определение}
Рассмотрим треугольник $ABC$, его медиану $BM$ и~биссектрису $BL$.
Тогда прямая, симметричная $BM$ относительно $BL$, называется \emph{симедианой}
треугольника.
Три симедианы пересекаются в~\emph{точке Лемуана}.
\end{claim}

\begin{problems}

\item
Касательные к~описанной окружности треугольника $ABC$ в~точках $A$ и~$C$
пересекаются в~точке~$P$.
Докажите, что прямая~$BP$ содержит симедиану~$BS$.

\item
Точка $S$~--- основание симедианы~$BS$ треугольника $ABC$.
Докажите соотношение $AS : CS = AB^2 : BC^2$.

\item
Свойства \emph{гармонического четырехугольника.}
\\
Следующие свойства вписанного четырехугольника $ABCD$ эквивалентны:
\begin{itemize}
    \item
    $AB \cdot CD = AD \cdot BC$.
    \item
    Касательные к~окружности в~точках $B$ и~$D$ пересекаются на~прямой~$AC$.
    \item
    Прямая~$AC$~--- симедиана треугольника $ABD$.
    \item
    $\angle AMB = \angle DCB$, где $M$~--- середина диагонали~$AC$.
    \item
    $\angle AMB = \angle AMD$, где $M$~--- середина диагонали~$AC$.
\end{itemize}
Докажите, что какие-то
\\
\subproblem 3;
\qquad
\subproblem 4;
\qquad
\subproblem 5
\\
утверждений равносильны.

\item
Высоты $AA_{1}$ и~$CC_{1}$ остроугольного треугольника $ABC$ пересекаются
в~точке~$H$.
Точка~$B_{0}$~--- середина стороны~$AC$.
Докажите, что точка пересечения прямых, симметричных $BB_{0}$ и~$HB_{0}$
относительно биссектрис углов $ABC$ и~$AHC$ соответственно, лежит
на~прямой~$A_{1}C_{1}$.

\item
В~остроугольном треугольнике $ABC$ высоты $AA_{1}$, $BB_{1}$ и~$CC_{1}$
пересекаются в~точке~$H$.
Из~$H$ провели перпендикуляры к~прямым $B_{1}C_{1}$ и~$A_{1}C_{1}$, которые
пересекли лучи $CA$ и~$CB$ в~точках $P$ и~$Q$ соответственно.
Докажите, что перпендикуляр из~точки~$C$ к~прямой $A_{1}B_{1}$ проходит через
середину~$PQ$.

\item
Биссектриса угла $ABC$ пересекает $AC$ и~окружность~$\omega_{1}$, описанную
около треугольника $ABC$, в~точках $D$ и~$E$ соответственно.
Окружность~$\omega_{2}$, построенная на~отрезке~$DE$ как на~диаметре,
пересекает окружность~$\omega_{1}$ в~точках $E$ и~$F$.
Докажите, что прямая, симметричная прямой~$BF$ относительно прямой~$BD$,
содержит медиану треугольника $ABC$.

%\item
%В~параллелограмме $ABCD$ точка~$M$, такая что $\angle MAD = \angle MCD$.
%Докажите, что $\angle MBA = \angle MDA$.

\item
Вписанная окружность~$\omega$ треугольника $ABC$ касается стороны~$BC$
в~точке~$D$.
Прямая~$AD$ пересекает $\omega$ в~точке $L \neq D$.
Точка~$K$~--- центр вневписанной окружности треугольника $ABC$, касающейся
стороны~$BC$.
Точки $M$ и~$N$~--- середины отрезков $BC$ и~$KM$ соответственно.
Докажите, что точки $B$, $C$, $N$~и~$L$ лежат на~одной окружности.

\end{problems}


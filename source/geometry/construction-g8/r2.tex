% $date: 2018-06-13
% $timetable:
%   g8r2:
%     2018-06-13:
%       2:

\worksheet*{Построение циркулем и~линейкой}

% $authors:
% - Ольга Дмитриевна Телешева

\textbf{Простейшие построения:}
\begin{itemize}

\item
на~прямой отложить от~данной точки отрезок заданной длины;

\item
отложить от~данного луча в~данную полуплоскость угол, равный данному углу;

\item
построить серединный перпендикуляр к~данному отрезку;

\item
разделить данный угол пополам;

\item
из~данной точки прямой восставить перпендикуляр к~данной прямой;

\item
из~данной точки вне прямой опустить перпендикуляр на~эту прямую;

\item
построить прямую, параллельную данной, проходящей через заданную точку;

\item
построить треугольник по~трем сторонам.

\end{itemize}


\textbf{Обозначения в~треугольнике $ABC$:}
\\
$a, b, c$~--- стороны треугольника;
\\
$m_{a}, m_{b}, m_{c}$~--- медианы треугольника;
\\
$h_{a}, h_{b}, h_{c}$~--- высоты треугольника;
\\
$P$~--- периметр треугольника;
\\
$R$~--- радиус описанной окружности треугольника;
\\
$r$~--- радиус вписанной окружности треугольника.


\begin{problems}

\item\begin{tabbing}
\rlap{Постройте треугольник $ABC$ по}
\hspace{0.30\linewidth}\=
\hspace{0.30\linewidth}\=
\\
\subproblem $\angle A=90^o, h_{a}, a$;
\>
\subproblem $R, a, b$;
\>
\subproblem $a, h_{b}, m_{a}$;
\\
\subproblem $h_{a}, h_{b}, \angle C$;
\>
\subproblem $a, b, m_{c}$;
\>
\subproblem $m_{a}, m_{b}, c$;
\\
\subproblem $m_{b}$ и отрезкам, на которые высота $h_{a}$ делит сторону $a$;
\\
\subproblem $\angle A, \angle B, r$;
\>
\subproblem $\angle A, \angle B, P$;
\>
\subproblem $m_{a}, m_{b}, m_{c}$.
\end{tabbing}

%\item
%На~плоскости нарисована окружность.
%Постройте ее центр.

%\item
%Постройте окружность данного радиуса, проходящую через две данные точки.

\item
На~плоскости нарисована окружность.
Через данную точку проведите к~ней касательную.

\item
Постройте окружность данного радиуса, проходящую через данную точку
и~касающуюся данной прямой.

\item
Постройте окружность, проходящую через данную точку и~касающуюся двух данных
параллельных прямых.

\item
Постройте точку $M$ внутри данного треугольника так, что
$S_{ABM} : S_{BCM} : S_{ACM} = 1 : 2 : 3$.

\end{problems}


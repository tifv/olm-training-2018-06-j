% $date: 2018-06-02
% $timetable:
%   g9r2:
%     2018-06-02:
%       2:

\worksheet*{Задача №255. Добавка}

% $authors:
% - Артемий Алексеевич Соколов

\begin{problems}

\item
К~двум окружностям проведены общая внешняя и~общая внутренняя касательные.
Докажите, что прямая, соединяющая две точки касания на~первой окружности
и~прямая, соединяющая две точки касания на~второй окружности, пересекаются
на~линии центров этих окружностей.

\item
Одна из~вневписанных окружностей треугольника $ABC$ касается стороны~$AB$
и~продолжений сторон $CA$ и~$CB$ в~точках $C_{1}$, $B_{1}$ и~$A_{1}$
соответственно, другая касается стороны~$AC$ и~продолжений сторон $BA$ и~$BC$
в~точках $B_{2}$, $C_{2}$ и~$A_{2}$ соответственно.
\\
\subproblem
Прямые $A_{1}B_{1}$ и~$A_{2}B_{2}$ пересекаются в~точке~$P$,
а~прямые $A_{1}C_{1}$ и~$A_{2}C_{2}$~--- в~точке~$Q$.
Докажите, что прямая~$PQ$ проходит через вершину~$A$.
\\
\subproblem
Прямые $A_{1}B_{1}$ и~$A_{2}C_{2}$ пересекаются в~точке~$X$,
а~прямые $A_{1}C_{1}$ и~$A_{2}B_{2}$~--- в~точке~$Y$.
Докажите, что прямая~$XY$ проходит через вершину~$A$.
\\
\subproblem
Докажите, что отрезок~$AY$ равен радиусу вписанной окружности
треугольника $ABC$.

\item
В~треугольнике $ABC$ выполняется равенство $3 AC = AB + BC$.
Вписанная в~треугольник окружность касается сторон $AB$ и~$BC$
в~точках $K$ и~$L$ соответственно;
$DK$ и~$EL$~--- ее диаметры.
Докажите, что точки пересечения прямых $AE$ и~$CD$ с~прямой~$KL$ равноудалены
от~середины отрезка~$AC$.

\item
В~треугольнике $ABC$ ($AB > AC$) проведены биссектриса~$AL$ и~медиана~$AM$.
Прямая, проведенная через $M$ параллельно стороне~$AB$, пересекает $AL$
в~точке~$D$.
Прямая, проведенная через $L$ параллельно стороне $AC$, пересекает $AM$
в~точке~$E$.
Докажите, что $ED \perp AD$.

\item
Вневписанная в~треугольник $ABC$ окружность с~центром~$I_{A}$
(соответствующая вершине~$A$) касается его сторон/продолжений сторон
в~точках $A_{1}$, $B_{1}$, $C_{1}$.
Пусть $P$ и~$Q$~--- точки пересечения пар прямых
$A_{1}C_{1}$ и~$CI_{A}$, $A_{1}B_{1}$ и~$BI_{A}$.
$AP$ и~$AQ$ пересекают $BC$ в~точках $M$, $N$.
Докажите, что точка~$A_{1}$~--- середина $MN$.

\item
Вписанная в~треугольник $ABC$ окружность имеет центр~$I$, касается его
сторон $BC$, $CA$, $AB$ в~точках $A_{1}$, $B_{1}$, $C_{1}$ соответственно;
$X$~--- проекция $A_{1}$ на~$B_{1}C_{1}$.
Докажите, что прямая, соединяющая середины отрезков $A_{1}X$ и~$B_{1}C_{1}$,
проходит через ортоцентр треугольника $BIC$.

\end{problems}


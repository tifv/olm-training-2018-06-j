% $date: 2018-06-02
% $timetable:
%   g9r1:
%     2018-06-02:
%       1:

\worksheet*{Задача №255}

% $authors:
% - Артемий Алексеевич Соколов

\begin{problems}

\item
Пусть $P$~--- проекция вершины~$B$ треугольника $ABC$ биссектрису угла~$C$,
$I$~--- центр вписанной окружности.
Докажите утверждения:
\\
\subproblem
$P$ лежит на~средней линии, параллельной стороне~$AC$;
\\
\subproblem
$P$ лежит на~прямой, соединяющей точки касания вписанной в~треугольник
окружности со~сторонами $AB$, $AC$;
\\
\subproblem
$P$ лежит на~прямой, соединяющей точки касания вневписанной в~треугольник
окружности со~стороной~$AB$ и~продолжением стороны~$AC$.

\item
Пусть $P$ и~$Q$~--- проекции вершин $A$ и~$C$ неравнобедренного
треугольника $ABC$ на~биссектрисы углов при вершинах $C$, $A$ соответственно.
А~еще $A_{1}$~--- середина стороны~$BC$, $C_{1}$~--- середина стороны~$AB$.
Докажите, что прямые $PC_{1}$ и~$QA_{1}$ пересекаются на~стороне~$AC$.

\item
Вневписанная окружность касается продолжений сторон $BA$ и~$BC$
в~точках~$N$ и~$M$.
Биссектриса угла смежного с~$\angle A$ пересекает $NM$ в~точке~$K$.
Докажите, что $CK \perp AK$.

\item
На~плоскости выбран фиксированный угол с~вершиной~$O$ и~фиксирована точка~$A$
на~одной его стороне ($A \neq O$).
Пусть $B$~--- произвольная точка на~другой стороне угла ($B \neq O$),
а~$P$ и~$Q$~--- точки касания вписанной в~треугольник $AOB$ окружности
со~сторонами соответственно $OB$ и~$AB$.
Докажите, что все такие прямые~$PQ$ проходят через одну точку.

\item
На~стороне~$AB$ прямоугольной трапеции $ABCD$
($AD \parallel BC$, $AB \perp BC$) нашлась такая точка~$X$ что
$\angle AXD = \angle BXC$.
Пусть $Y$~--- проекция точки~$X$ на~прямую~$CD$.
Докажите, что описанная окружность треугольника $ABY$ проходит через
середину~$CD$.

\item
Окружность, вписанная в~прямоугольный треугольник $ABC$ с~гипотенузой~$AB$,
касается его сторон $BC$, $AC$ и~$AB$ в~точках $A_{1}$, $B_{1}$ и~$C_{1}$
соответственно.
Пусть $B_{1}H$~--- высота треугольника $A_{1}B_{1}C_{1}$.
Докажите, что точка~$H$ лежит на~биссектрисе угла $CAB$.

\item
В~треугольнике $ABC$ ($AB > AC$) проведены биссектриса~$AL$ и~медиана~$AM$.
Прямая, проведенная через $M$ параллельно стороне~$AB$, пересекает $AL$
в~точке~$D$.
Прямая, проведенная через $L$ параллельно стороне $AC$, пересекает $AM$
в~точке~$E$.
Докажите, что $ED \perp AD$.

\item
Вневписанная в~треугольник $ABC$ окружность с~центром~$I_{A}$
(соответствующая вершине~$A$) касается его сторон/продолжений сторон
в~точках $A_{1}$, $B_{1}$, $C_{1}$.
Пусть $P$ и~$Q$~--- точки пересечения пар прямых
$A_{1}C_{1}$ и~$CI_{A}$, $A_{1}B_{1}$ и~$BI_{A}$.
$AP$ и~$AQ$ пересекают $BC$ в~точках $M$, $N$.
Докажите, что точка~$A_{1}$~--- середина $MN$.

\end{problems}


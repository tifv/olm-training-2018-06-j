% $date: 2018-06-04
% $timetable:
%   g8r1:
%     2018-06-04:
%       2:
%   g8r2:
%     2018-06-04:
%       1:

\worksheet*{Перегибая бумагу, получаем задачу}

% $authors:
% - Александр Давидович Блинков

\begin{problems}

\item
\begin{minipage}[t][][t]{0.65\linewidth}
У~листа бумаги только один ровный край.
Лист согнули, а~потом разогнули обратно.
$A$~--- общая точки линии сгиба и~ровного края (рисунок справа).
Постройте перпендикуляр к~этой линии в~точке~$A$, используя только перегибание
бумаги.
\end{minipage}\hfill
\begin{minipage}[t][][b]{0.32\linewidth}\vspace{-2ex}\flushright
\jeolmfigure{1}
\vspace{1ex}
\end{minipage}

\item
Бумажный прямоугольник согнули по~диагонали, а~затем, сложив еще два раза,
получили четырехслойный треугольник.
Найдите угол между стороной и~диагональю исходного прямоугольника.

\item
\begin{minipage}[t][][t]{0.65\linewidth}
Квадратный лист бумаги сначала сложили вдвое, а~затем так, как показано
на~рисунке справа.
Чему равен отмеченный угол?
\end{minipage}\hfill
\begin{minipage}[t][][b]{0.32\linewidth}\vspace{-1ex}\flushright
\jeolmfigure{3-1}
\qquad
\jeolmfigure{3-2}
\vspace{1ex}
\end{minipage}

\item
\begin{minipage}[t][][t]{0.70\linewidth}
Прямоугольный лист бумаги согнули, совместив вершину с~серединой
противоположной короткой стороны (рисунок справа).
Оказалось, что треугольники I и~II равны.
Найдите длинную сторону прямоугольника, если короткая равна 8.
\end{minipage}\hfill
\begin{minipage}[t][][b]{0.27\linewidth}\vspace{-1ex}\flushright
\jeolmfigure{4}
\vspace{1ex}
\end{minipage}

\item
\begin{minipage}[t][][t]{0.62\linewidth}
Два угла прямоугольного листа бумаги согнули так, как показано на~рисунке
справа.
Противоположная сторона при этом оказалась разделенной на~три равные части.
Докажите, что закрашенный треугольник~--- равносторонний.
\end{minipage}\hfill
\begin{minipage}[t][][b]{0.35\linewidth}\vspace{-1ex}\flushright
\jeolmfigure{5}
\vspace{1ex}
\end{minipage}

\item
Прямоугольный лист бумаги перегнули по~прямой так, что противоположные вершины
совместились.
В~результате получились три треугольника: в~середине~--- один двухслойный,
а~по~краям~--- два однослойных.
Докажите, что двухслойный треугольник~--- равнобедренный.

\item
Бумажный прямоугольник $ABCD$ перегибается так, что точка $C$ попадает
в~точку~$C'$~--- середину стороны~$AD$.
Найдите отношение $DK : AB$, где $K$~--- точка линии сгиба на~стороне~$CD$.

\pagebreak[3]

\item
Из~квадратного листа бумаги сложили треугольник $MAN$ (рисунки ниже).
Найдите угол $ANM$.
\begin{center}
\jeolmfigure{8-1}
\qquad
\jeolmfigure{8-2}
\end{center}

\item
\begin{minipage}[t][][t]{0.70\linewidth}
Бумажный равносторонний треугольник перегнули по~прямой так, что одна из~вершин
попала на~противоположную сторону (рисунок справа).
Докажите, что углы двух получившихся белых треугольников соответственно равны.
\end{minipage}\hfill
\begin{minipage}[t][][b]{0.27\linewidth}\vspace{-1ex}\flushright
\jeolmfigure{9}
\vspace{1ex}
\end{minipage}

\item
Из~листа бумаги, одна сторона которого желтая, а~другая~--- белая, Саша вырезал
равнобедренный треугольник.
Затем он перегнул его по~биссектрисе угла при основании и~склеил.
Получился треугольник, состоящий из~двух равнобедренных треугольников: белого
и~желтого.
Докажите, что после перегибания у~него вновь получился равнобедренный
треугольник.

\item
Возьмём бумажный квадрат $ABCD$.
Пусть $M$~--- середина~$CD$.
Согнем квадрат так, чтобы сгиб проходил через точку~$M$, а~середина~$AD$ попала
на~диагональ~$AC$.
Разогнем лист, отрезок сгиба обозначим $MX$.
Теперь согнем квадрат так, чтобы сгиб снова проходил через точку~$M$,
а~середина~$BC$ попала на~диагональ~$BD$.
Разогнем лист, отрезок сгиба обозначим $MY$.
Докажите, что треугольник $MXY$~--- равносторонний.

\item
Петя вырезал из~бумаги прямоугольник, наложил на~него такой~же прямоугольник
и~склеил их по~периметру.
В~верхнем прямоугольнике он провел диагональ, опустил на~нее перпендикуляры
из~двух остальных вершин, после чего разрезал верхний прямоугольник по~этим
линиям и~отогнул полученные треугольники во~внешнюю сторону так, что вместе
с~нижним прямоугольником они образовали еще один прямоугольник.
Каким образом по~полученному прямоугольнику восстановить исходный, используя
только циркуль и~линейку?

\end{problems}


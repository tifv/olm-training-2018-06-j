% $date: 2018-06-03
% $timetable:
%   g9r1:
%     2018-06-03:
%       2:
%   g9r2:
%     2018-06-05:
%       1:

\worksheet*{Изогональное сопряжение}

% $authors:
% - Артемий Алексеевич Соколов

\begin{claim}{Определение}
Дан треугольник $ABC$.
Две точки $P$ и~$Q$ называются \emph{изогонально сопряженными} относительно
треугольника $ABC$, если
прямые $PA$ и~$QA$ симметричны относительно биссектрисы угла~$A$,
прямые $PB$ и~$QB$ симметричны относительно биссектрисы угла~$B$,
а~прямые $PC$ и~$QC$ симметричны относительно биссектрисы угла~$C$.
\end{claim}

\begin{problems}

\item
\subproblem
Докажите, что точка пересечения высот и~центр описанной окружности треугольника
изогонально сопряжены.
\\
\subproblem
Какие точки изогонально сопряжены самим себе?
\\
\subproblem
Во~что переходят точки описанной окружности при изогональном сопряжении?
\\
\subproblem
Касательные к~описанной окружности треугольника $ABC$ в~точках $B$ и~$C$
пересекаются в~точке~$P$.
Точка~$Q$ симметрична точке~$A$ относительно середины отрезка~$BC$.
Докажите, что точки $P$ и~$Q$ изогонально сопряжены.

\item
Опустим из~точки~$P$ перпендикуляры на~стороны треугольника
(или их продолжения) и~рассмотрим окружность, проходящую через основания
перпендикуляров.
Докажите, что эта окружность совпадает с~окружностью, построенной таким~же
образом для точки~$Q$, изогонально сопряженной точки~$P$.

%\item
%Докажите, что при изогональном сопряжении окружность, проходящая через вершины
%$B$ и~$C$, отличная от~описанной, переходит в~окружность, проходящую
%через $B$ и~$C$.

%\item
%Высоты $AA_{1}$ и~$CC_{1}$ остроугольного треугольника $ABC$ пересекаются
%в~точке~$H$.
%Точка~$B_{0}$~--- середина стороны~$AC$.
%Докажите, что точка пересечения прямых, симметричных $BB_{0}$ и~$HB_{0}$
%относительно биссектрис углов $ABC$ и~$AHC$ соответственно, лежит
%на~прямой~$A_{1}C_{1}$.

\item
В~трапеции $ABCD$ боковая сторона~$CD$ перпендикулярна основаниям,
$O$~--- точка пересечения диагоналей.
На~описанной окружности треугольника $OCD$ взята точка~$S$, диаметрально
противоположная точке~$O$.
Докажите, что $\angle BSC = \angle ASD$.

\item
$AA_{0}$, $BB_{0}$, $CC_{0}$~--- высоты треугольника $ABC$,
$H$~--- ортоцентр,
$M$~--- произвольная точка.
$A_{1}$ -- точка, симметричная $M$ относительно $BC$;
аналогично определим точки $B_{1}$, $C_{1}$.
Докажите, что прямые $A_{0}A_{1}$, $B_{0}B_{1}$, $C_{0}C_{1}$ пересекаются
в~одной точке.

\item
Стороны треугольника $ABC$ видны из~точки~$T$ под углами $120^{\circ}$.
Докажите, что прямые, симметричные прямым $AT$, $BT$ и~$CT$ относительно
прямых $BC$, $CA$ и~$AB$ соответственно, пересекаются в~одной точке.

%\item
%Чевианы $AA_{1}$, $BB_{1}$ и~$CC_{1}$ треугольника $ABC$ пересекаются
%в~точке~$P$, лежащей внутри треугольника.
%Известно, что $PA_{1} = PB_{1} = PC_{1}$.
%Докажите, что перпендикуляры, восставленные в~точках $A_{1}$, $B_{1}$ и~$C_{1}$
%к~сторонам треугольника $ABC$, пересекаются в~одной точке.

\item
Касательные к~описанной окружности остроугольного неравнобедренного
треугольника $ABC$, восстановленные в~вершинах $B$~и~$C$, пересекаются
в~точке~$P$.
Точка~$Q$~--- отражение основания высоты из~вершины $A$ относительно середины
стороны~$BC$.
Перпендикуляр к~прямой~$PQ$, восстановленный в~точке~$Q$, пересекает
прямые $AB$ и~$AC$ в~точках $B'$~и~$C'$ соответственно.
Докажите, что $\angle B'PB = \angle C'PC$.

\end{problems}


% $date: 2018-06-07
% $timetable:
%   g8r1:
%     2018-06-07:
%       2:

\worksheet*{Площади}

% $authors:
% - Ольга Дмитриевна Телешева

Каждой фигуре~$M$ на~плоскости сопоставим число~$S_{M}$, называемое
\emph{площадью}, обладающее следующими свойствами:
\\
\emph{(1)}\enspace
площадь неотрицательна;
\\
\emph{(2)}\enspace
площади равных фигур равны;
\\
\emph{(3)}\enspace
площадь объединения фигур, не~имеющих общих внутренних точек, есть сумма
площадей фигур;
\\
\emph{(4)}\enspace
площадь прямоугольника равна произведению его сторон.

\claim{Факт 1}
Пол в~комнате площадью~$S$ покрыт линолеумом общей площадью~$S$ так, что нет
участков, покрытых более чем в~два слоя.
Тогда площадь пола, покрытая дважды, равна площади пола, не~покрытой ни~разу.

\begin{problems}

\item
Дан параллелограмм $ABCD$.
\\
\subproblem
На~прямой~$AD$ взята точка~$M$.
Площадь треугольника $BMC$ равна $S$.
Какова площадь параллелограмма?
(Разберите все случаи).
\\
\subproblem
Пусть теперь точка~$M$ взята внутри параллелограмма и~соединена со~всеми
вершинами.
Общая площадь двух треугольников, примыкающих к~сторонам $AD$ и~$BC$,
равна $S$.
Найдите площадь параллелограмма.

\end{problems}

\claim{Факт 2}
Даны две параллельные прямые $AB$ и~$l$.
Тогда площадь треугольника $ABC$ не~зависит от~от~выбора точки~$C$, находящейся
на~прямой~$l$.

\begin{problems}

\item
Выразите $S_{ABC}$ через $S$.
\begin{center}
\hfill
    \jeolmfigure{../figures/m-1}
\hfill
    \jeolmfigure{../figures/m-2}
\hfill
    \jeolmfigure{../figures/m-3}
\hfill
\null\par
\hfill
    \jeolmfigure{../figures/m-4}
\hfill
    \jeolmfigure{../figures/m-5}
\hfill
\null
\end{center}

%\item
%В~трапеции $ABCD$ с~меньшим основанием~$BC$ через точку~$B$ проведена прямая,
%параллельная $CD$ и~пересекающая диагональ~$AC$ в~точке~$E$.
%Сравните площади треугольников $ABC$ и~$DEC$.

%\item
%Докажите, что в~трапеции с~проведенными диагоналями треугольники, прилежащие
%к~боковым сторонам, равновелики.

\item\jeolmlabel{/geometry/area-g8/r1/:problem:rectangle}%
Докажите, что площадь темно-серой части
(рис.~\jeolmref{/geometry/area-g8/r1/:problem:rectangle:fig})
равна сумме площадей светло-серых частей.

\begin{figure}[ht]
\begin{center}
    \jeolmfigure{../figures/rectangle}
    \caption{к задаче~\jeolmref{/geometry/area-g8/r1/:problem:rectangle}}
    \jeolmlabel{/geometry/area-g8/r1/:problem:rectangle:fig}
\end{center}
\end{figure}

\item
Через точку~$D$, лежащую на~стороне~$BC$ треугольника $ABC$, проведены прямые,
параллельные двум другим сторонам и~пересекающие $AB$ и~$AC$ соответственно
в~точках $E$ и~$F$.
Докажите, что $S_{CDE} = S_{BDF}$.

\item
Точка~$E$~--- середина стороны~$BC$ треугольника $ABC$.
Точки $D$ и~$G$ на~сторонах $BC$ и~$AC$ соответственно таковы, что
$AD \perp BC$ и~$GE \perp BC$.
Докажите, что $S_{CGD} = S_{DGAB}$.

\begin{figure}[ht]
\begin{center}
    \jeolmfigure{../figures/pythagoras}
    \caption{к задаче~\jeolmref{/geometry/area-g8/r1/:problem:pythagoras}}
    \jeolmlabel{/geometry/area-g8/r1/:problem:pythagoras:fig}
\end{center}
\end{figure}

\item\jeolmlabel{/geometry/area-g8/r1/:problem:pythagoras}
На~сторонах прямоугольного треугольника $ABC$ построены 3 квадрата
(рис.~\jeolmref{/geometry/area-g8/r1/:problem:pythagoras:fig}).
Докажите, что одноцветные площади попарно равны.
\\
Итак, мы доказали \emph{теорему Пифагора:} сумма квадратов катетов равна
квадрату гипотенузы.

\item
В~шестиугольнике $ABCDEF$ диагонали $AD$, $BE$ и~$CF$ пересекаются
в~одной точке и~точкой пересечения делятся пополам.
Докажите, что площадь шестиугольника $ABCDEF$ равна удвоенной площади
треугольника $ACE$.

\item
Дан выпуклый четырехугольник $ABCD$.
Через середину~$G$ диагонали~$BD$ проведена прямая, параллельная
диагонали~$AC$, пересекающая сторону~$DC$ в~точке $H$.
Докажите, что отрезок~$AH$ делит площадь четырехугольника $ABCD$ пополам.

%\item
%Докажите, что медианы разбивают треугольник на~шесть равновеликих
%треугольников.

%\item
%Существует~ли треугольник на~клетчатой бумаге со~сторонами большими, чем 10,
%площадь которого равна половине площади одной клетки?

%\item
%Диагонали выпуклого четырехугольника делят его на~4 части, площади которых,
%взятые последовательно, равны $S_1$, $S_2$, $S_3$, $S_4$.
%Докажите, что $S_1 \cdot S_3 = S_2 \cdot S_4$.

\item
Внутри правильного треугольника выбрали точку.
Докажите, что сумма расстояний от~нее до~сторон треугольника от~данного выбора
не~зависит.
А~верно~ли это для правильного пятиугольника?

%\item
%Середины соседних сторон выпуклого четырехугольника соединены отрезками.
%Докажите, что площадь полученного четырехугольника вдвое меньше площади
%данного.

\item
На~сторонах треугольника $ABC$ взяты точки $P$, $Q$ и~$R$, делящие его стороны
в~отношениях $BP : PC = p$, $CQ : QA = q$, $AR : RB = r$.
Чему равно отношение площадей треугольников $PQR$ и~$ABC$?

%\item
%Многоугольник периметра~$P$ описан около окружности радиуса~$r$.
%Найдите площадь многоугольника.

%\item
%Могут~ли длины высот треугольника равняться 1, 2 и~3?

\item
Каждая из~сторон выпуклого четырехугольника разделена на~три равные части,
и~соответствующие точки противоположных сторон соединены.
Докажите, что площадь центрального четырехугольника в~девять раз меньше площади
целого.

\item
Каждая сторона выпуклого $n$-угольника $A_{1}A_{2}{\ldots}A_{n}$ продолжена
на~свою длину так, что точка $A_{i}$~--- середина отрезка $A_{i-1} A'_{i}$,
$A_{1}$~--- середина отрезка $A_{n}A'_{1}$.
Площадь исходного многоугольника равна $S$.
Найдите площадь полученного многоугольника $A'_{1}A'_{2}{\ldots}A'_{n}$.

\end{problems}


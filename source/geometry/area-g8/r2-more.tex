% $date: 2018-06-11
% $timetable:
%   g8r2:
%     2018-06-11:
%       1:

\worksheet*{Площади. Добавка}

% $authors:
% - Ольга Дмитриевна Телешева

\begin{problems}

\item
Докажите, что на~рисунке слева площадь темно-серой части равна сумме площадей
светло-серых частей.

\begin{center}
\hfill
    \jeolmfigure{../figures/n-1}
\hfill
    \jeolmfigure{../figures/n-2}
\hfill
\null
\end{center}

\item
Докажите, что на~рисунке справа сумма площадей темно-серых частей равна сумме
площадей светло-серых частей.

%\item
%Пусть площадь треугольника $ABC$ равна $S$.
%Найдите площадь треугольника, образованного медианами этого треугольника.

\item
Диагонали выпуклого четырехугольника делят его на~4 части, площади которых,
взятые последовательно, равны $S_1$, $S_2$, $S_3$, $S_4$.
Докажите, что $S_1 \cdot S_3 = S_2 \cdot S_4$.

\item
На~сторонах треугольника $ABC$ взяты точки $P$, $Q$ и~$R$, делящие его стороны
в~отношениях $BP : PC = p$, $CQ : QA = q$, $AR : RB = r$.
Чему равно отношение площадей треугольников $PQR$ и~$ABC$?

\item
Каждая из~сторон выпуклого четырехугольника разделена на~три равные части,
и~соответствующие точки противоположных сторон соединены.
Докажите, что площадь центрального четырехугольника в~девять раз меньше площади
целого.

\item
Каждая сторона выпуклого $n$-угольника $A_{1}A_{2}{\ldots}A_{n}$ продолжена
на~свою длину так, что точка $A_{i}$~--- середина отрезка $A_{i-1} A'_{i}$,
$A_{1}$~--- середина отрезка $A_{n}A'_{1}$.
Площадь исходного многоугольника равна $S$.
Найдите площадь полученного многоугольника $A'_{1}A'_{2}{\ldots}A'_{n}$.

\end{problems}


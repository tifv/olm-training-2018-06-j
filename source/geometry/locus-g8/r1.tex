% $date: 2018-06-08
% $timetable:
%   g8r1:
%     2018-06-08:
%       2:

\worksheet*{ГМТ}

% $authors:
% - Ольга Дмитриевна Телешева

\claim{Определение}
\emph{Геометрическое место точек} (сокращенно \emph{ГМТ}), обладающих некоторым
свойством,~--- это множество всех точек, для которых выполнено указанное
свойство.

\claim{Простейшие примеры}\resetsubproblem
\\
\subproblem
ГМТ, удаленных от~данной точки на~расстояние $R > 0$,~--- окружность;
\\
\subproblem
ГМТ, равноудаленных от~концов отрезка,~--- серединный перпендикуляр к~этому
отрезку;
\\
\subproblem
ГМТ, равноудаленных от~лучей данного угла,~--- биссектриса этого угла;
\\
\subproblem
ГМТ, из~которых данный отрезок виден под прямым углом,~--- окружность без двух
точек;

%\claim{Упражнение 1}
%Даны точки $A$, $B$, $C$, не~лежащие на~одной прямой.
%Найдите ГМТ $M$ таких, что
%\\
%\subproblem
%прямая~$CM$ не~пересекает отрезок~$AB$;
%\quad
%\subproblem
%отрезок~$CM$ пересекает отрезок~$AB$.

\claim{Упражнение}
Докажите, что серединные перпендикуляры к~сторонам треугольника пересекаются
в~одной точке~--- центре описанной окружности.

\begin{problems}

\item
Дан отрезок~$AB$.
Найдите геометрическое место точек~$C$ таких, что
%\\
%\subproblem $\angle BAC = 17^{\circ}$;
\\
\subproblem $AC + BC = AB$;
\\
\subproblem $AC < BC$;
\\
\subproblem
расстояние от~$C$ до~какой-нибудь из~двух точек $A$ и~$B$ меньше длины
отрезка~$AB$;
\\
\subproblem
точка~$C$~--- центр окружности, проходящей через точки $A$ и~$B$.
\\
\subproblem
Высота~$CH$ треугольника $ABC$ равна $h$;
\\
\subproblem
медиана~$CM$ треугольника $ABC$ равна $m$;
\\
\subproblem
треугольник $ABC$~--- равнобедренный;
\\
\subproblem
треугольник $ABC$~--- остроугольный;
\\
\subproblem
треугольник $ABC$~--- тупоугольный;
\\
\subproblem
$\angle B$~--- второй по~величине угол неравнобедренного треугольника $ABC$.
% то~же самое, что
%Дан треугольник $ABC$.
%Найдите ГМТ $X$, удовлетворяющих неравенствам $AX \leq BX \leq CX$.

%\item
%Даны две параллельные прямые.
%Найдите ГМТ, для которых расстояние до~первой вдвое больше, чем до~второй.

%\item
%Дан квадрат $ABCD$.
%Найдите ГМТ, которые расположены ближе к~центру квадрата, чем к~любой его
%вершине.

\item
Дан квадрат $ABCD$.
Найдите ГМТ таких, что сумма расстояний от~каждой из~них до~прямых $AB$ и~$CD$
равна сумме расстояний до~прямых $BC$ и~$AD$.

\item
Найдите геометрическое место внутренних точек прямоугольника $ABCD$ таких, что
$S_{ABM} + S_{CDM} = S_{ADM} + S_{BCM}$.

\item
\subproblem\jeolmlabel{/geometry/locus-g8/r1/:problem:bisector}%
Найдите ГМТ, равноудаленных от~двух данных прямых.
\\
\subproblem
Выведите из пункта~\jeolmsubref{/geometry/locus-g8/r1/:problem:bisector},
что три биссектрисы треугольника пересекаются в~одной точке~---
\emph{центре вписанной окружности.}
\\
\subproblem
Выведите из пункта~\jeolmsubref{/geometry/locus-g8/r1/:problem:bisector},
что биссектриса внутреннего угла и~две биссектрисы внешних углов
треугольника пересекаются в~одной точке~---
\emph{центре вневписанной окружности.}

\item
\subproblem\jeolmlabel{/geometry/locus-g8/r1/:problem:median}%
Дан треугольник $ABC$.
Найдите ГМТ $X$ таких, что $S_{ABX} = S_{CBX}$.
\\
\subproblem
Выведите из пункта~\jeolmsubref{/geometry/locus-g8/r1/:problem:median},
что медианы треугольника пересекаются в~одной точке.

\item
\subproblem\jeolmlabel{/geometry/locus-g8/r1/:problem:altitude}%
Дан треугольник $ABC$.
Найдите ГМТ $X$ таких, что $AX^2 - XC^2 = AB^2 - BC^2$.
\\
\subproblem
Выведите из пункта~\jeolmsubref{/geometry/locus-g8/r1/:problem:altitude},
что высоты треугольника пересекаются в~одной точке.

%\item
%Лестница стоит вплотную к~стене, а~к~ее %середине прилипла муха.
%Лестница плавно скользит, и~в~итоге падает %на~пол
%(ее верхний конец всегда прижат к~стене).
%Какова траектория мухи?

\item
Для выпуклого четырехугольника $ABCD$ верно $AB < BC$ и~$AD < DC$.
Точка~$M$ лежит на~диагонали~$BD$.
Докажите, что $AM < MC$.

\item
Отрезок постоянной длины движется по~плоскости так, что его концы скользят по
\\
\subproblem сторонам прямого угла;
\qquad
\subproblem окружности.
\\
По~какой траектории движется середина этого отрезка?

\item
На~плоскости даны точки $A$ и~$B$ и~прямая $\ell$.
По~какой траектории движется точка пересечения медиан треугольников $ABC$, если
точка~$C$ движется по~прямой $\ell$?

\item
Точки $A$ и~$B$ движутся по~окружности так, что хорда~$AB$ всегда проходит
через фиксированную точку~$M$ внутри окружности.
Найдите ГМТ, являющихся серединами отрезка~$AB$.

%\item
%На~плоскости даны два непересекающихся круга.
%Найдите ГМТ $M$, удовлетворяющих такому условию: каждая прямая, проходящая
%через точку~$M$, пересекает хотя~бы один из~этих кругов.

\end{problems}


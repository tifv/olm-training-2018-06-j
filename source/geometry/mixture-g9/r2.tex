% $date: 2018-06-12
% $timetable:
%   g9r2:
%     2018-06-12:
%       2:
%   g9r3:
%     2018-06-12:
%       1:

\worksheet*{Разнобой-повторение}

% $authors:
% - Артемий Алексеевич Соколов

\begin{problems}

\item
Дан прямоугольник $ABCD$ и~точка~$P$.
Прямые, проходящие через $A$ и~$B$ и~перпендикулярные, соответственно,
$PC$ и~$PD$, пересекаются в~точке~$Q$.
Докажите, что $PQ \perp AB$.
% Паскаль

\item
Даны треугольник $ABC$ и~некоторая точка~$T$.
Пусть $P$ и~$Q$~--- основания перпендикуляров, опущенных из~точки~$T$
на~прямые $AB$ и~$AC$ соответственно, а~$R$ и~$S$~--- основания
перпендикуляров, опущенных из~точки~$A$ на~прямые $TC$ и~$TB$ соответственно.
Докажите, что точка пересечения прямых $PR$ и~$QS$ лежит на~прямой~$BC$.
% Паскаль

\item
Даны окружность, ее хорда~$AB$ и~точка~$W$~--- середина меньшей дуги~$AB$.
На~большей дуге~$AB$ выбирается произвольная точка~$C$.
Касательная к~окружности из~точки~$C$ пересекает касательные из~точек $A$ и~$B$
в~точках $X$ и~$Y$ соответственно.
Прямые $WX$ и~$WY$ пересекают прямую~$AB$ в~точках $N$ и~$M$ соответственно.
Докажите, что длина отрезка~$NM$ не~зависит от~выбора точки~$C$.

\item
К~двум окружностям $\omega_{1}$ и~$\omega_{2}$, пересекающимся
в~точках $A$ и~$B$, проведена их общая касательная~$CD$ ($C$ и~$D$~--- точки
касания соответственно, точка~$B$ ближе к~прямой~$CD$, чем $A$).
Прямая, проходящая через $A$, вторично пересекает $\omega_{1}$ и~$\omega_{2}$
в~точках $K$ и~$L$ соответственно ($A$ лежит между $K$ и~$L$).
Прямые $KC$ и~$LD$ пересекаются в~точке~$P$.
Докажите, что $PB$~--- симедиана треугольника $KPL$.
% изогональное сопряжение + симедиана

\item
Неравнобедренный треугольник $ABC$ вписан в~окружность~$\omega$.
Касательная к~этой окружности в~точке~$C$ пересекает прямую~$AB$ в~точке~$D$.
Пусть $I$~--- центр вписанной окружности треугольника $ABC$.
Прямые $AI$ и~$BI$ пересекают биссектрису угла $CDB$ в~точках $Q$ и~$P$
соответственно.
Пусть $M$~--- середина отрезка~$PQ$.
Докажите, что прямая~$MI$ проходит через середину дуги $ACB$
окружности~$\omega$.
% симедиана

\item
В~треугольнике $ABC$ проведены высоты $AA_{1}$ и~$BB_{1}$
и~биссектрисы $AA_{2}$ и~$BB_{2}$;
вписанная окружность касается сторон $BC$ и~$AC$ в~точках $A_{3}$ и~$B_{3}$.
Докажите, что прямые $A_{1}B_{1}$, $A_{2}B_{2}$, $A_{3}B_{3}$ пересекаются
в~одной точке или параллельны.
% 255 + Паскаль

\end{problems}


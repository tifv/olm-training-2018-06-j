% $date: 2018-06-11
% $timetable:
%   g9r1:
%     2018-06-11:
%       2:

\worksheet*{Разнобой-повторение. Добавка}

% $authors:
% - Артемий Алексеевич Соколов

\begin{problems}

\item
Даны окружность, ее хорда~$AB$ и~точка~$W$~--- середина меньшей дуги~$AB$.
На~большей дуге~$AB$ выбирается произвольная точка~$C$.
Касательная к~окружности из~точки~$C$ пересекает касательные из~точек $A$ и~$B$
в~точках $X$ и~$Y$ соответственно.
Прямые $WX$ и~$WY$ пересекают прямую~$AB$ в~точках $N$ и~$M$ соответственно.
Докажите, что длина отрезка~$NM$ не~зависит от~выбора точки~$C$.

\item
Дан треугольник $ABC$ и~точка~$F$ такая, что
$\angle AFB = \angle BFC = \angle CFA$.
Прямая, проходящая через $F$ и~перпендикулярная $BC$, пересекает медиану,
проведенную из~вершины~$A$, в~точке~$A_{1}$.
Точки $B_{1}$ и~$C_{1}$ определяются аналогично.
Докажите, что $A_{1}$, $B_{1}$ и~$C_{1}$ являются тремя вершинами правильного
шестиугольника, три другие вершины которого лежат на~сторонах
треугольника $ABC$.

\item
В~треугольнике $ABC$ касательные к~описанной окружности в~точках $A$ и~$C$
пересекаются в~точке~$P$.
Прямая~$BP$ повторно пересекает описанную окружность в~точке~$D$.
Прямая, проходящая через точку~$D$ и~основание высоты треугольника
из~вершины~$B$, повторно пересекает описанную окружность в~точке~$K$.
Докажите, что точка~$K$, середина стороны~$AC$ и~ортоцентр лежат
на~одной прямой.

\item
Окружность~$\omega$ пересекает
сторону~$BC$ треугольника $ABC$ в~точках $A_{1}$, $A_{2}$,
сторону~$AC$ в~точках $B_{1}$, $B_{2}$,
сторону $AB$ в~точках $C_{1}$, $C_{2}$,
причем порядок точек
$B {-} C_{1} {-} C_{2} {-} A$,
$A {-} B_{1} {-} B_{2} {-} C$,
$C {-} A_{1} {-} A_{2} {-} B$.
Прямые $A_{1}C_{1}$ и~$A_{2}B_{2}$ пересекаются в~точке~$A_{3}$,
$B_{1}A_{1}$ и~$B_{2}C_{2}$ пересекаются в~точке~$B_{3}$,
$C_{1}B_{1}$ и~$C_{A}B_{2}$ пересекаются в~точке~$C_{3}$.
Докажите, что прямые $AA_{3}$, $BB_{3}$, $CC_{3}$ пересекаются в~одной точке.

\end{problems}


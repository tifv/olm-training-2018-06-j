% $date: 2018-06-02
% $timetable:
%   g9r3:
%     2018-06-02:
%       1:

\worksheet*{Конечные множества. Отображение конечных множеств}

% $authors:
% - Александр Савельевич Штерн

\begingroup
    \def\abs#1{\lvert #1 \rvert}%
    \def\binom#1#2{\mathrm{C}_{#1}^{#2}}%

%\epigraph{
%  Центральной задачей комбинаторики можно считать задачу размещения объектов
%  в~соответствии со~специальными правилами и~нахождения числа способов,
%  которыми это может быть сделано.}{М.\, Холл}

\subsection*{Обсуждение}

\begin{exercises}

\item
Обозначения: объединение, пересечение, дополнение, порядок (число элементов)
конечного множества.

\item
Пусть $A$, $B$~--- два конечных множества.
Отображение из~множества~$A$ в~множество~$B$~--- это правило, которое каждому
элементу множества~$A$ сопоставляет элемент множества~$B$.
\\
Отображение называется \textit{инъективным}, если образы любых различных
элементов различны.
\\
Отображение называется \textit{сюрьективным}, если у~каждого элемента
множества~$B$ есть прообраз.
\\
Отображение называется \textit{биективным}, если оно инъективно и~сюрьективно
одновременно.
Биективное отображение между двумя конечными множествами можно установить тогда
и~только тогда, когда в~них поровну элементов.

\item\emph{Упражнение.}
(Правило произведения считаем известным).
\\
Пусть $\abs{A} = n$.
Найдите:
\\
\subproblem
количество отображений из~$A$ в~$A$ и~количество биективных отображений
из~$A$ в~$A$;
\\
\subproblem
количество инъективных отображений из~множества $\{1, 2, \ldots, m\}$
в~множество~$A$;
\\
\subproblem
количество отображений из~$A$ в~$\{0, 1\}$ и~количество сюрьективных
отображений из~$A$ в~$\{0, 1\}$;
\\
\subproblem
количество $k$-элементных подмножеств в~$A$;
\\
\subproblem
количество четноэлементных подмножеств в~$A$ и~количество нечетноэлементных
подмножеств в~$A$.

\item
Задача о~подсчете подмножеств.
Биномиальные коэффициенты и~некоторые биномиальные тождества.
\\
Симметрия: $\binom{n}{k} = \binom{n}{n-k}$.
\\
Внесение-вынесение: $k \cdot \binom{n}{k} = n \cdot \binom{n-1}{k-1}$.
\\
Сложение-разложение: $\binom{n}{k} = \binom{n-1}{k} + \binom{n-1}{k-1}$.
\\
Трином:
$\binom{m}{k} \cdot \binom{n}{m} = \binom{n}{k} \cdot \binom{n-k}{m-k}$.
\\
Свертка Вандермонда:
$\binom{r+s}{n} = \sum_{k} \binom{r}{k} \cdot \binom{s}{n-k}$.

\item
Задача о~шарах и~перегородках (шары неразличимы, ящики пронумерованы).

\item
Формула включений-исключений.
Задача о~числе сюрьекций.

\end{exercises}


\subsection*{Задачи}

\begin{problems}

\item
Сколько можно составить перестановок из~$n$ элементов, в~которых данные
$m$~элементов не~стоят рядом?

\item
Пусть $A$, $B$, $C$~--- количество подмножеств $n$-элементного множества,
в~которых количеств элементов имеет вид $3k$, $3k+1$, $3k+2$ соответственно.
Могут~ли все эти три числа быть одинаковыми?
Все три разными?

\item
Найдите сумму
\(
    \binom{n}{1} + 2 \binom{n}{2} + 3 \binom{n}{3}
    + \ldots +
    n \binom{n}{n}
\).

\item
Найдите сумму
\(
    \binom{n}{1} + 2^{2} \binom{n}{2} + 3^{2} \binom{n}{3}
    + \ldots +
    n^{2} \binom{n}{n}
\).

\item
Докажите, что при любом натуральном~$n$ число $\binom{2n}{n}$
делится на~$n + 1.$

\item
Докажите, что при любых натуральных $0 < k < m < n$ числа
$\binom{n}{k}$ и~$\binom{n}{m}$ имеют общий делитель, отличный от~$1$.

\item
Найдите количество отображений из~множества $\{1, 2, \ldots, n\}$ в~это~же
множество, при которых
элемент $1$ имеет $n_{1}$ прообразов,
элемент $2$ имеет $n_{2}$ прообразов, \ldots,
элемент $k$ имеет $n_{k}$ прообразов
(а~остальные элементы прообразов не~имеют).

\item
Докажите тождество
\(
    \sum
        \dfrac{n!}{n_{1}! \cdot n_{2}! \cdot \ldots \cdot n_{k}!}
=
    k^{n}
\),
где суммирование производится по~всем упорядоченным разбиениям числа~$n$
на~$k$~слагаемых.

\item
В~множестве, состоящем из~$n$~элементов, выбрано $2^{n-1}$ подмножеств, каждые
три из~которых имеют общий элемент.
Докажите, что все эти подмножества имеют общий элемент.

\end{problems}

\endgroup % \def\abs \def\binom


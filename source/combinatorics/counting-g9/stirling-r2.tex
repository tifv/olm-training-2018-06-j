% $date: 2018-06-04
% $timetable:
%   g9r2:
%     2018-06-04:
%       1:

\worksheet*{Числа Стирлинга и~числа Белла}

% $authors:
% - Александр Савельевич Штерн

\begingroup
    \def\StirlingN#1#2{\mathrm{S}_{#1}^{#2}}%
    \def\BellN#1{\mathrm{B}_{#1}}%

\subsection*{Обсуждение}

\begin{itemize}

\item
Сколькими способами можно разбить 4-эле\-мент\-ное множество на~два
подмножества?

\item
Сколькими способами можно разбить 7-эле\-мент\-ное множество на~4 подмножества?

\end{itemize}

\emph{Число Стирлинга} $\StirlingN{n}{k}$
равно количеству способов, которыми можно разбить $n$-эле\-мент\-ное множество
на~$k$ подмножеств.

\subsection*{Упражнения}

\begin{exercises}

\item
Найдите $\StirlingN{n}{1}, \StirlingN{n}{2}, \StirlingN{n}{n}$.

\item
Найдите $\StirlingN{5}{3}$.

\end{exercises}

\claim{Подсказка}
Пусть четыре элемента вы уже расположили.
Как можно подложить к~ним пятый?
\par
Рекуррентное соотношение:
$\StirlingN{n}{k} = \StirlingN{n-1}{k-1} + k \StirlingN{n-1}{k}$.

\begin{exercises}

\itemy{3}
Постройте несколько первых строк треугольника Стирлинга.

\end{exercises}


\subsection*{Обсуждение}

\begin{itemize}

\item
Сколькими способами можно разбить 4-элементное множество на~непересекающиеся
подмножества?

\item
Сколькими способами можно разбить 7-элементное множество на~непересекающиеся
подмножества?

\end{itemize}

\emph{Число Белла} $\BellN{n}$ равно количеству способов, которыми можно
разбить $n$-элементное множество на~непересекающиеся подмножества
(возможно одно):
\[
    \BellN{n}
=
    \sum_{i=1}^{n}
        \StirlingN{n}{i}
\, . \]
%Найдите $\BellN{7}$.
Последовательность Белла:
1, 1, 2, 5, 15, 52, 203, 877, 4140, 21\,147, 115\,975, \ldots


\subsection*{Задачи}

\begin{problems}

\item
Найдите число всех сюрьективных отображений $n$-элементного множества
на~$k$-эле\-мен\-тное.
Попытайтесь из~этих соображений получить явную (пусть и~очень нудную) формулу
для чисел Стирлинга.

\item
Обозначим следующий многочлен $[x]_{k} = x (x - 1) (x - 2) \ldots (x - k + 1)$.
Докажите равенство
\[
    x^{n} = \sum_{k} \alpha(n, k) [x]_{k}
\]
и~найдите коэффициенты $\alpha(n, k)$.

\item
Могут~ли три последовательных числа Белла быть нечетными?

\item
Найдите количество отображений из~множества $\{1, 2, \ldots, n\}$ в~это~же
множество, при которых
элемент $1$ имеет $n_{1}$ прообразов,
элемент $2$ имеет $n_{2}$ прообразов, \ldots,
элемент $k$ имеет $n_{k}$ прообразов
(а~остальные элементы прообразов не~имеют).

\item
Назовем \emph{лестницей высоты~$n$} фигуру, состоящую из~всех клеток
квадрата $n \times n$, лежащих не~выше диагонали.
Сколькими различными способами можно разбить лестницу высоты $n$ на~несколько
прямоугольников, стороны которых идут по~линиям сетки, а~площади попарно
различны?

\end{problems}

\endgroup % \def\StirlingN \def\BellN


% $date: 2018-06-01
% $timetable:
%   g9r1:
%     2018-06-01:
%       1:
%   g9r2:
%     2018-06-01:
%       2:

\worksheet*{Конечные множества. Отображение конечных множеств}

% $authors:
% - Александр Савельевич Штерн

\begingroup
    \def\binom#1#2{\mathrm{C}_{#1}^{#2}}%

%\epigraph{
%  Центральной задачей комбинаторики можно считать задачу размещения объектов
%  в~соответствии со~специальными правилами и~нахождения числа способов,
%  которыми это может быть сделано.}{М.\, Холл}

%\subsection*{Обсуждение}
%
%\begin{enumerate}
%
%\item
%Обозначения: объединение, пересечение, дополнение, порядок (число элементов)
%конечного множества.
%
%\item
%Перестановки.
%Подсчет числа биекций на~конечном множестве.
%
%\item
%Другие задачи о~функциях.
%Подсчет числа отображений вида $f \colon A \to \{1, 2\}$ и~подсчет числа
%подмножеств.
%
%\item
%Сюрьекции и~инъекции.
%Подсчет числа инъективных отображений $f \colon \{1, 2, \ldots, m\} \to A$.
%
%\item
%Задача о~подсчете подмножеств.
%Биномиальные коэффициенты и~некоторые биномиальные тождества.
%\\
%Симметрия: $\binom{n}{k} = \binom{n}{n-k}$.
%\\
%Внесение-вынесение: $k \cdot \binom{n}{k} = n \cdot \binom{n-1}{k-1}$.
%\\
%Сложение-разложение: $\binom{n}{k} = \binom{n-1}{k} + \binom{n-1}{k-1}$.
%\\
%Трином:
%$\binom{m}{k} \cdot \binom{n}{m} = \binom{n}{k} \cdot \binom{n-k}{m-k}$.
%\\
%Свертка Вандермонда:
%$\binom{r+s}{n} = \sum_{k} \binom{r}{k} \cdot \binom{s}{n-k}$.
%
%\item
%Задача о~шарах и~перегородках (шары неразличимы, ящики пронумерованы).
%
%\item
%Формула включений-исключений.
%Задача о~числе сюрьекций.
%
%\end{enumerate}


%\subsection*{Задачи}

\begin{problems}

\item
Сколько можно составить перестановок из~$n$ элементов, в~которых данные
$m$~элементов не~стоят рядом?

\item
Докажите, что при любом натуральном~$n$ число $\binom{2n}{n}$ делится на~$n + 1$.

\item
Докажите, что при любых натуральных $0 < k < m < n$ числа
$\binom{n}{k}$ и~$\binom{n}{m}$ имеют общий делитель, отличный от~$1$.

\item
Найдите сумму
\(
    \binom{n}{1} + 2 \binom{n}{2} + 3 \binom{n}{3}
    + \ldots +
    n \binom{n}{n}
\).

\item
Найдите сумму
\(
    \binom{n}{1} + 2^{2} \binom{n}{2} + 3^{2} \binom{n}{3}
    + \ldots +
    n^{2} \binom{n}{n}
\).

\item
Докажите тождество
\(
    \sum
        \dfrac{n!}{n_{1}! \cdot n_{2}! \cdot \ldots \cdot n_{k}!}
=
    k^{n}
\),
где суммирование производится по~всем упорядоченным разбиениям числа~$n$
на~$k$~слагаемых.

\item
В~множестве~$A$ зафиксирован набор подмножеств $A_{1}, A_{2}, \ldots, A_{n}$.
Разрешается брать пересечения и~объединения имеющихся множеств, а~также их
дополнения до~множества~$A$.
Сколько различных подмножеств множества~$A$ можно получить таким способом?

\item
Докажите, что число $\dfrac{(2m)!(2n)!}{m!n!(n+m)!}$~--- целое.

\item
Для любой упорядоченной пары подмножеств $A$, $B$ (не~обязательно различных)
$n$-эле\-ментного множества подсчитано число элементов
в~пересечении $A \cap B$, а~затем все эти числа просуммированы.
Чему равна полученная сумма?

\end{problems}

\endgroup % \def\binom


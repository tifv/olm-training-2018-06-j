% $date: 2018-06-05
% $timetable:
%   g9r3:
%     2018-06-05:
%       1:

\worksheet*{Формула включений-исключений}

% $authors:
% - Александр Савельевич Штерн

\begingroup
    \def\abs#1{\lvert #1 \rvert}%
    \def\binom#1#2{\mathrm{C}_{#1}^{#2}}%

Через $\abs{A}$ обозначается число элементов конечного множества~$A$.
Для произвольного набора конечных множеств $A_{1}, A_{2}, \ldots, A_{n}$
справедлива формула
\begin{align*}
    \abs{A_{1} \cup A_{2} \cup \ldots \cup A_{n}}
& =
    \abs{A_{1}} + \abs{A_{2}} + \ldots + \abs{A_{n}}
    - \abs{A_{1} \cap A_{2}} - \abs{A_{1} \cap A_{3}}
    - \ldots \\ & \mspace{20mu} \ldots
    - \abs{A_{n-1} \cap A_{n}}
    + \ldots +
    (-1)^{n-1} \abs{A_{1} \cap A_{2} \cap \ldots \cap A_{n}}
\, . \end{align*}

%\subsection*{Задачи}

\begin{problems}

\item
Сколькими способами можно переставить буквы в~слове <<тартар>> так, чтобы
нашлись две одинаковые буквы, стоящие рядом?

\item
Сколько существует различных натуральных чисел, меньших миллиона, в~запись
которых входит каждая из~цифр $1$, $2$, $3$, $4$?

\item
Антон, Артем и~Вера решили вместе 100 задач по~математике.
Каждый из~них решил 60~задач.
Назовем задачу \emph{трудной,} если ее решил только один человек, и~легкой,
если ее решили все трое.
Насколько отличается количество трудных задач от~количества легких?

\item
\subproblem
По~пустыне идет караван из~8 верблюдов.
Найти число способов, которыми можно переставить верблюдов так, чтобы ни~один
из~верблюдов не~стоял на~старом месте?
\\
\subproblem
Обозначим через $D_{n}$ число перестановок на~множестве чисел от~1 до~$n$,
в~которых ни~одно из~чисел не~совпадает со~своим номером.
Найдите формулы для величин $D_{n}$.
Числа $D_{n}$ называются \emph{субфакториалами.}

\item\emph{Задача о~счастливых билетах.}
Автобусный билет представляет собой произвольный набор шести цифр.
Билет называется \emph{счастливым,} если сумма первых трех его цифр равна сумме
трех последних.
\\
\subproblem
Доказать, что число счастливых билетов равно числу билетов с~суммой цифр 27.
\\
\subproblem
Сколько счастливых билетов существует?

\item
\subproblem
По~пустыне идет караван из~8 верблюдов.
Найти число способов, которыми можно переставить верблюдов так, чтобы ни~один
из~верблюдов не~шел непосредственно за~тем верблюдом, за~которым шел раньше.
\\
\subproblem
Обозначим через $E_{n}$~--- число перестановок, в~которых никакие два
последовательных числа (большее за~меньшим) не~идут друг за~другом.
Найти формулы для величины $E_{n}$.

\item
На~кафтане площадью~$1$ размещены $5$~заплат, площадь каждой из~которых
не~меньше $1/2$.
Докажите, что найдутся две заплаты, площадь общей части которых
не~меньше $1/5$.

\item
Пусть $m$ и~$n$~--- натуральные числа, $m < n$.
Найдите сумму
\[
    \binom{n}{1} (n - 1)^{m}
    - \binom{n}{2}(n-2)^{m}
    + \ldots +
    (-1)^{n-1} \binom{n}{n-1}
\, . \]

\end{problems}

\endgroup % \def\abs \def\binom


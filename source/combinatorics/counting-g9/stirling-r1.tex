% $date: 2018-06-03
% $timetable:
%   g9r1:
%     2018-06-03:
%       1:

\worksheet*{Числа Стирлинга и~числа Белла}

% $authors:
% - Александр Савельевич Штерн

\begingroup
    \def\binom#1#2{\mathrm{C}_{#1}^{#2}}%
    \def\StirlingN#1#2{\mathrm{S}_{#1}^{#2}}%
    \def\BellN#1{\mathrm{B}_{#1}}%

\subsection*{Обсуждение}

\begin{itemize}

\item
Сколькими способами можно разбить 4-эле\-мент\-ное множество на~два
подмножества?

\item
Сколькими способами можно разбить 7-эле\-мент\-ное множество на~4 подмножества?

\end{itemize}

\emph{Число Стирлинга} $\StirlingN{n}{k}$
равно количеству способов, которыми можно разбить $n$-эле\-мент\-ное множество
на~$k$ подмножеств.

\subsection*{Упражнения}

\begin{exercises}

\item
Найдите $\StirlingN{n}{1}, \StirlingN{n}{2}, \StirlingN{n}{n}$.

\item
Найдите $\StirlingN{5}{3}$.

\end{exercises}

\claim{Подсказка}
Пусть четыре элемента вы уже расположили.
Как можно подложить к~ним пятый?
\par
Рекуррентное соотношение:
$\StirlingN{n}{k} = \StirlingN{n-1}{k-1} + k \StirlingN{n-1}{k}$.

\begin{exercises}

\itemy{3}
Постройте несколько первых строк треугольника Стирлинга.

\end{exercises}


\subsection*{Обсуждение}

\begin{itemize}

\item
Сколькими способами можно разбить 4-элементное множество на~непересекающиеся
подмножества?

\item
Сколькими способами можно разбить 7-элементное множество на~непересекающиеся
подмножества?

\end{itemize}

\emph{Число Белла} $\BellN{n}$ равно количеству способов, которыми можно
разбить $n$-элементное множество на~непересекающиеся подмножества
(возможно одно):
\[
    \BellN{n}
=
    \sum_{i=1}^{n}
        \StirlingN{n}{i}
\, . \]

\begin{exercises}

\itemy{4}
Найдите $\BellN{7}$.

\end{exercises}


\subsection*{Задачи}

\begin{problems}

\item
Докажите рекуррентное соотношение:
\[
    \BellN{n + 1} = \sum_{i=0}^{n} \binom{n}{i} \BellN{i}
\, . \]

\item
Найдите число всех сюрьективных отображений $n$-элементного множества
на~$k$-эле\-мен\-тное.

\item
В~$n$-элементном множестве выделили несколько подмножеств, содержащих
$x_{1}, x_{2}, \ldots, x_{k}$ элементов соответственно.
Известно, что ни~одно из~этих множеств не~содержится в~другом.
Докажите неравенство
\[
    \frac{1}{\binom{n}{x_{1}}} + \frac{1}{\binom{n}{x_{2}}}
    + \ldots +
    \frac{1}{\binom{n}{x_{n}}}
\leq
    1
. \]

\item
Обозначим следующий многочлен $[x]_{k} = x (x - 1) (x - 2) \ldots (x - k + 1)$.
Докажите равенство
\[
    x^{n} = \sum_{k} \alpha(n, k) [x]_{k}
\]
и~найдите коэффициенты $\alpha(n, k)$.

\item
Назовем \emph{лестницей высоты~$n$} фигуру, состоящую из~всех клеток
квадрата $n \times n$, лежащих не~выше диагонали.
Сколькими различными способами можно разбить лестницу высоты $n$ на~несколько
прямоугольников, стороны которых идут по~линиям сетки, а~площади попарно
различны?

\item
Каждый элемент треугольника Лейбница равен сумме элементов, стоящих под ним,
а~числа на~границах строк фиксированы.
Найдите разумную формулу для элемента, стоящего на~$k$-м месте строки
с~номером~$n$.
\[ \arraycolsep=2pt
    \begin{array}{ccccccccccccc}
        & & & & & \frac{1}{1} \\
        & & & & \frac{1}{2} & & \frac{1}{2} \\
        & & & \frac{1}{3} & & \frac{1}{6} & & \frac{1}{3} \\
        & & \frac{1}{4} & & \frac{1}{12} & & \frac{1}{12} & & \frac{1}{4} \\
        & \frac{1}{5} & & \frac{1}{20} & & \frac{1}{30} &
            & \frac{1}{20} & & \frac{1}{5} \\
        \frac{1}{6} & & \frac{1}{30} & & \frac{1}{60} &
            & \frac{1}{60} & & \frac{1}{30} & & \frac{1}{6}
    \end{array}
\]

\item
Докажите равенство
\[
    \frac{1}{x (x + 1) (x + 2) \ldots (x + n)}
=
    \sum_{k=0}^{n}
        \alpha(n, k) \frac{1}{x + k}
\]
и~найдите коэффициенты $\alpha(n, k)$.

\end{problems}

\endgroup % \def\binom \def\StirlingN \def\BellN


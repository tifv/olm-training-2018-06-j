% $date: 2018-06-13
% $timetable:
%   g9r2:
%     2018-06-13:
%       2:

\worksheet*{Дополнительные задачи по~комбинаторике}

% $authors:
% - Андрей Юрьевич Кушнир

\begin{problems}

%\item
%Докажите, что на~ребрах любого графа можно расставить стрелки так, чтобы для
%каждой вершины модуль разности входящей и~исходящей ее степени
%не~превосходил $1$.

\item
Последовательность $a_{n}$ натуральных чисел называется \emph{полиномиальной,}
если существует многочлен $P(x)$ с~целыми коэффициентами такой, что
$a_{n} = P(n)$ при всех натуральных $n$.
Можно~ли раскрасить натуральные числа в~два цвета так, чтобы не~было
одноцветных полиномиальных последовательностей?

\item
Рассмотрим перестановку
$\sigma = (\sigma_{1}, \sigma_{2}, \ldots, \sigma_{70})$
чисел $1, 2, \ldots, 70$.
За~один ход разрешается поменять два числа местами.
Нашей целью является получить набор $(1, 2, \ldots, 70)$
из~$\sigma$.
Обозначим через $N({\sigma})$ минимальное число ходов, которые приводят к~цели
из~перестановки $\sigma$.
Найдите максимально возможное значение $N({\sigma})$ по~всем
$\sigma \in S_{70}$.
%(Так как это перестановка, среди чисел
%$\sigma_{1}, \sigma_{2}, \ldots, \sigma_{70}$ нет одинаковых.)

%\item
%На~координатной плоскости отмечены некоторые точки с~целыми координатами.
%Известно, что внутри любого круга радиуса $2018$ хотя~бы одна точка отмечена.
%Докажите, что существует окружность, проходящая хотя~бы через четыре отмеченные
%точки.

\item
Стрелки полного ориентированного графа раскрашены в~два цвета.
Докажите, что в~нем существует вершина, от~которой можно добраться до~любой
другой вершины по~некоторому монохромному пути одного из~двух цветов.

%\item
%В~$100$ ящиках лежат яблоки, апельсины и~бананы.
%Докажите, что можно так выбрать $51$~ящик, что в~них окажется не~менее половины
%всех яблок, не~менее половины всех апельсинов и~не~менее половины всех бананов.

\end{problems}


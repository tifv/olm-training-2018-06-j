% $date: 2018-06-04
% $timetable:
%   g9r1:
%     2018-06-04:
%       1:
%   g9r2:
%     2018-06-03:
%       1:

\worksheet*{Вариация}

% $authors:
% - Андрей Юрьевич Кушнир

\begin{problems}

\item
На~отрезке~$AB$ отмечено $2n$ различных точек, симметричных относительно
середины $AB$.
При этом $n$ из~них покрашены в~красный цвет, оставшиеся $n$~--- в~синий.
Докажите, что сумма расстояний от~точки~$A$ до~красных точек равна сумме
расстояний от~точки~$B$ до~синих точек.

\item
Сумма нескольких натуральных чисел равна $2017$.
Найдите максимально возможное их произведение.

%\item
%Имеется три кучи камней.
%Сизиф таскает по~одному камню из~кучи в~кучу.
%За~каждое перетаскивание он получает от~Зевса количество монет, равное разности
%числа камней в~куче, в~которую он кладет камень, и~числа камней в~куче,
%из~которой он берет камень
%(сам перетаскиваемый камень при этом не~учитывается).
%Если указанная разность отрицательна, то~Сизиф возвращает Зевсу соответствующую
%сумму.
%(Если Сизиф не~может расплатиться, то~великодушный Зевс позволяет ему совершать
%перетаскивание в~долг.)
%В~некоторый момент оказалось, что все камни лежат в~тех~же кучах, в~которых
%лежали первоначально.
%Каков наибольший суммарный заработок Сизифа на~этот момент?
%% Прибережем до~лучших времен.

\item
В~однокруговом турнире по~теннису участвовало $2n+1$ человек:
$p_{1}, p_{2}, \ldots, p_{2n+1}$
(каждый сыграл с~каждым ровно один раз, ничьих не~бывает).
Обозначим через~$w_{i}$ число побед игрока~$p_{i}$.
Найдите максимум и~минимум (в~зависимости от~$n$) величины
$w_{1}^2 + w_{2}^2 + \ldots + w_{2n+1}^2$.

%\item
%На~прямой отмечены $2n$ различных точек, при этом $n$ из~них покрашены
%в~красный цвет, остальные $n$~--- в~синий.
%Докажите, что сумма попарных расстояний между точками одного цвета
%не~превосходит суммы попарных расстояний между точками разного цвета.

\item
По~окружности расставлены несколько положительных чисел, не~превосходящих
единицы.
Докажите, что окружность можно разбить на~$2018$ дуг так, чтобы суммы чисел
на~соседних дугах отличались не~более чем на~один.
Если чисел на~дуге нет, то~сумма чисел на~дуге считается равной нулю.

\item
Председателю дачного кооператива необходимо распределить $49$ квадратных
участков (в~виде квадрата $7 \times 7$) среди $49$~дачников.
Каждый дачник враждует не~более чем с~$6$ другими.
Докажите, что это можно сделать так, чтобы никакие два враждующих дачника
не~получили соседние по~стороне участки.

\item
В~таблице $n \times m$ расставлены действительные числа так, что сумма чисел
в~каждом столбце и~каждой строке целая.
Докажите, что каждое число в~таблице можно заменить на~его верхнюю или нижнюю
целую часть так, чтобы сумма чисел в~каждом столбце и~каждой строке
не~изменилась.

\item
\emph{Хромой ладьей} назовем ладью, которая за~один ход может сдвинуться только
на~одну клетку.
Хромая ладья за~$64$ хода обошла все клетки шахматной доски и~вернулась
на~исходную клетку.
Докажите, что число ее ходов по~горизонтали не~равно числу ходов по~вертикали.

% про яблоки и бананы с зоналки задача еще подходит, но ее давал сильной группе
% на весенних сборах.
% Есть еще про додекаэдр от Кушнира, тоже давали уже.

\end{problems}


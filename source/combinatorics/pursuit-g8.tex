% $date: 2018-06-11
% $timetable:
%   g8r1:
%     2018-06-11:
%       1:
%   g8r2:
%     2018-06-11:
%       2:

\worksheet*{Догонялки на~плоскости}

% $authors:
% - Сергей Александрвич Дориченко

\subsection*{Конус Маха}

\begin{problems}

\item
В~поле проходит прямая дорога, по~ней со~скоростью 10\,км/ч едет автобус.
Укажите все точки поля, из~которых можно догнать автобус, если бежать
\\
\subproblem с~той~же скоростью;
\subproblem со~скоростью 5\,км/ч.

\item
Самолет, летящий в~два раза быстрее скорости звука, вылетел из~точки~$A$
и~летит в~точку~$B$ по~прямой.
Самолет непрерывно издает звук, который распространяется во~все стороны.
Нарисуйте все точки, до~которых успеет дойти звук самолета за~время, пока
самолет летит из~$A$ в~$B$.
(Считайте, что всё происходит в~плоскости.)

\item
В~поле проходит прямая дорога.
Человек, стоящий на~дороге в~точке~$A$, может идти по~полю со~скоростью
не~более 3\,км/ч и~по~дороге со~скоростью не~более 6\,км/ч.
Нарисуйте, куда он может попасть за~1\,ч.

\item
В~поле проходят две перпендикулярные друг другу прямые дороги.
Человек, стоящий на~перекрестке, может идти по~полю со~скоростью не~более
3\,км/ч и~по~дорогам со~скоростью не~более
\\
\subproblem 6 км/ч;
\qquad
\subproblem $3\sqrt2$ км/ч.
\\
Нарисуйте все точки, в~которые он может попасть за~1 час.

\item
Пункт~$A$ находится в~лесу в~5\,км от~прямой дороги, пункт~$B$~--- на~дороге,
расстояние от~$A$ до~$B$~--- 13\,км (по~полю).
Скорость пешехода на~дороге~--- 5\,км/ч, в~лесу~--- 3\,км/ч.
За~какое наименьшее время пешеход сможет попасть из~$A$~в~$B$?

\end{problems}

\subsection*{Найди стратегию}

\begin{problems}

\item
Миша стоит в~центре круглой лужайки радиуса 100\,м.
Каждую минуту он шагает на~1\,м, заранее объявляя, в~каком направлении хочет
шагнуть.
Катя имеет право заставить его сменить направление на~противоположное.
Может~ли Миша действовать так, чтобы когда-нибудь гарантированно выйти
с~лужайки?

\item
В~центре квадрата сидит заяц, в~каждом углу~--- волк.
Может~ли заяц выбежать из~квадрата, если волки бегают лишь по~сторонам квадрата
с~максимальной скоростью, которая больше максимальной скорости зайца
в~$1{,}4$ раза?

\item
На~плоскости играют волк и~несколько овец.
Сначала ходит волк, потом какая-нибудь овца, потом волк, потом опять
какая-нибудь овца, и~т.\,д.
И~волк, и~овцы передвигаются за~ход в~любую сторону не~более, чем на~1\,м.
Для любого~ли числа овец существует такая начальная позиция, что волк
не~поймает ни~одной~овцы?

\item
Город представляет собой бесконечную клетчатую плоскость
(линии~--- улицы, клеточки~--- кварталы).
На~одной из~улиц через каждые 100 кварталов на~перекрестках стоит
по~милиционеру.
Где-то в~городе есть бандит
(его местонахождение неизвестно, но~перемещается он только по~улицам).
Цель милиции~--- увидеть бандита.
Есть~ли у~милиции алгоритм наверняка достигнуть своей цели?
Максимальные скорости милиции и~бандита~--- какие-то конечные, но~неизвестные
нам величины (у~бандита скорость может быть больше, чем у~милиции).
Милиция видит вдоль улиц во~все стороны на~бесконечное расстояние.

\end{problems}

\subsection*{Ловим в~несколько этапов}

\begin{problems}

\item
На~плоскости играют Левша и~невидимая блоха.
За~ход Левша проводит прямую, а~блоха прыгает на~1\,м, не~пересекая ни~одной
прямой Левши.
Если блоха не~может сделать ход, она становится видимой и~сдается.
Может~ли Левша гарантированно выиграть?

\item
Некто угнал старую полицейскую машину, максимальная скорость которой
составляет 90\% от~максимальной скорости новой, и~едет по~бесконечной в~обе
стороны дороге.
Полицейский на~новой машине не~знает, ни~когда машину угнали (это могло
случиться сколь угодно давно), ни~в~каком направлении уехал угонщик.
Сможет~ли полицейский поймать~угонщика?

\itemx{*}
На~бесконечной клетчатой сетке (линии~--- улицы, клетки~--- кварталы) трое
полицейских ловят вора.
Местонахождение вора неизвестно, но~перемещается он только по~улицам.
Максимальные скорости у~полицейских и~вора одинаковы.
Вор считается пойманным, если он оказался на~одной улице с~полицейским.
Смогут~ли полицейские гарантированно поймать вора?
(Полицейские тоже движутся только по~улицам.)

\end{problems}


% $date: 2018-06-09
% $timetable:
%   g8r2:
%     2018-06-09:
%       2:

\worksheet*{Анализ информации~--- 2}

% $authors:
% - Сергей Александрович Дориченко

\begin{problems}

\item
В~деревне живут 64~хоббита, каждый~--- в~отдельном домике.
По~вечерам они ходят друг к~другу в~гости.
За~один вечер каждый хоббит посещает всех, кого можно застать дома, причем если
он идет в~гости, то~у~себя дома в~этот вечер уже не~появляется.
Могут~ли хоббиты действовать так, чтобы за~6 вечеров среди любых двух жителей
деревни хотя~бы один побывал в~гостях у~другого?

\item
Вам и~мне надевают на~голову шляпу.
Каждая шляпа либо черная, либо белая.
Вы видите мою шляпу, я~--- вашу, но~никто не~видит своей шляпы.
Каждый из~нас (не~подглядывая и~не~подавая друг другу никаких сигналов) должен
попытаться угадать цвет своей шляпы.
Для этого по~команде одновременно каждый называет цвет~--- <<черный>> или
<<белый>>.
Если хоть один угадал~--- мы выиграли.
Перед этим нам дали возможность посовещаться.
Как действовать, чтобы в~любой ситуации выиграть?

\item
\subproblem
Мудрецам предстоит испытание: им завяжут глаза, наденут каждому черный или
белый колпак, построят в~колонну и~развяжут глаза.
Затем мудрецы по~очереди, начиная с~последнего (который видит всех), будут
называть цвет своего колпака.
Кто ошибется~--- тому голову с~плеч.
Сколько мудрецов гарантированно может спастись?
(Каждый видит всех впереди стоящих;
у~мудрецов до~испытания есть время, чтобы договориться.)
\\
\subproblem
А~если колпаки могут быть $k$ данных цветов?
\\
\subproblemx{*}
А~если колпаки могут быть $100$ данных цветов, 100 мудрецов стоят по~кругу
(видят друг друга) и~называют цвета своих колпаков одновременно, и~нужно, чтобы
цвет своего колпака угадал хотя~бы один?
% мудрец?

\item
\subproblem
В~тюрьме 100 узников.
Надзиратель сказал им:
<<Я дам вам поговорить друг с~другом, а~потом рассажу по~отдельным камерам,
и~общаться вы уже не~сможете.
Иногда я буду одного из~вас отводить в~комнату, в~которой есть лампа (вначале
она выключена).
Уходя из~комнаты, можно оставить лампу как включенной, так
и~выключенной.
Если в~какой-то момент кто-то из~вас скажет мне, что вы все уже побывали в~комнате, и~будет прав, я всех выпущу на~свободу.
А~если неправ~--- скормлю всех крокодилам.
И~не~волнуйтесь, что кого-то забудут,~--- если будете молчать, то~все побываете
в~комнате, и~ни~для кого никакое посещение комнаты не~станет последним.>>
Придумайте стратегию, гарантирующую узникам освобождение.
\\
\subproblem
А~если неизвестно, была~ли лампа в~самом начале включена или нет?

\item
Секретный код к~любому из~сейфов ФБР~--- это целое число
\\
\subproblem от~1 до~900;
\qquad
\subproblemx{*} от~1 до~1700.
\\
Два шпиона узнали по~одному коду каждый и~решили обменяться информацией.
Согласовав заранее свои действия, они встретились на~берегу реки возле кучи
из~26 камней.
Сначала первый шпион кинул в~воду один или несколько камней, потом~--- второй,
потом~--- опять первый, и~т.\,д. до~тех пор, пока камни не~кончились.
Затем шпионы разошлись.
Каким образом могла быть передана информация?
(Шпионы не~сказали друг другу ни~слова.)

\item
Имеются 8 монет, семь из~которых одинаковые, а~одна фальшивая и~отличается
по~весу (неизвестно, в~какую сторону).
Также есть чашечные весы, которые показывают правильный результат, если
на~чашах разный вес, а~если вес одинаковый, то~вместо равенства показывают что
попало.
\\
\subproblem
Придумайте способ найти фальшивую монету и~узнать, тяжелее она настоящих или
легче.
\\
\subproblem
Можно~ли гарантированно найти фальшивую монету всего за~4 взвешивания?

\item
Фокуснику завязывают глаза, а~зритель выкладывает в~ряд $N$ одинаковых монет,
сам выбирая, какие~--- орлом вверх, а~какие~--- решкой.
Ассистент фокусника просит зрителя написать на~листе бумаги любое число от~$1$
до~$N$ и~показать всем присутствующим.
Увидев число, ассистент
указывает зрителю на~одну из~монет ряда и~просит
перевернуть ее.
Затем фокуснику развязывают глаза, он смотрит на~ряд монет и~безошибочно
определяет написанное зрителем число.
\\
\subproblem
Докажите, что если у~фокусника с~ассистентом есть способ, позволяющий фокуснику
гарантированно отгадывать число для $N = a$, то~есть способ и~для $N = 2 a$.
\\
\subproblem
Найдите все $N$, для которых у~фокусника с~ассистентом есть способ.

\item
Ботанический определитель использует 100 признаков.
Каждый признак либо есть у~растения, либо нет.
Определитель <<хороший>>, если любые два растения в~нем отличаются более чем
по~50 признакам.
Может~ли хороший определитель описывать более 50 растений?

\end{problems}


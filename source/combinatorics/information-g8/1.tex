% $date: 2018-06-07
% $timetable:
%   g8r1:
%     2018-06-07:
%       1:
%   g8r2:
%     2018-06-07:
%       2:

\worksheet*{Анализ информации~--- 1}

% $authors:
% - Сергей Александрович Дориченко

\begin{problems}

\item
Загадано натуральное число от~1 до~100.
Можно задавать вопросы, на~которые дается ответ <<да>> или <<нет>>.
За~какое наименьшее число вопросов всегда можно отгадать число, если
\\
\subproblem
каждый следующий вопрос задается после того, как получен ответ на~предыдущий
вопрос;
\\
\subproblem
надо заранее сказать все вопросы?

\item
В~каждую клетку доски $8 \times 8$ записано целое число от~$1$ до~$64$
(каждое по~разу).
За~один вопрос, указав любую совокупность полей, можно узнать числа, стоящие
на~этих полях
(без указания, какую клетку какое число занимает).
За~какое наименьшее число вопросов всегда можно узнать, какие числа где стоят?

\item
Туристы взяли в~поход $80$ банок консервов, веса которых известны и~различны
(есть список).
Вскоре надписи на~банках стерлись
%стали нечитаемыми,
и~только завхоз знает, где что.
Он хочет доказать всем, что в~какой банке находится, не~вскрывая банок
и~пользуясь только списком и~двухчашечными весами со~стрелкой, показывающей
разницу весов на~чашах.
Хватит~ли ему для этого
\\
\subproblem четырех;
\quad
\subproblem трех
\\
взвешиваний?

\item
Имеется 1000 бутылок с~вином, в~одной вино испорчено, и~10 белых мышей.
Если мышь выпьет плохого вина, то~через минуту станет фиолетовой.
Разрешается один раз накапать каждой мыши вина из~разных бутылок, дать им
выпить одновременно и~подождать минуту.
Как найти испорченное вино?

\item
Обезьяна хочет определить, из~окна какого самого низкого этажа 15-этажного дома
нужно бросить кокосовый орех, чтобы он разбился.
У~нее есть
\\
\subproblem 1 орех;
\qquad
\subproblem 2 ореха.
\\
Какого наименьшего числа бросков ей заведомо хватит?
(Неразбившийся орех можно бросать снова.)

\item
\subproblem
Двое показывают карточный фокус.
Первый снимает пять карт из~колоды, содержащей 52 карты
(предварительно перетасованной кем-то из~зрителей),
смотрит в~них и~после этого выкладывает их в~ряд слева направо, причем одну
из~карт кладет рубашкой вверх, а~остальные~--- картинкой вверх.
Второй участник фокуса отгадывает закрытую карту.
Докажите, что они могут так договориться, что второй всегда будет угадывать
карту.
\\
\subproblem
Та~же задача, но~первый %участник
выкладывает слева направо четыре карты картинкой вверх, а~одну не~выкладывает.
Второй должен угадать невыложенную карту.

\item
\subproblem
В~жюри олимпиады 11 человек.
Материалы олимпиады хранятся в~сейфе.
Какое наименьшее число замков должен иметь сейф, чтобы можно было изготовить
сколько-то ключей и~так их раздать членам жюри, чтобы доступ в~сейф был
возможен если и~только если соберется не~менее 6 членов жюри?
\\
\subproblem
Круглая арена цирка освещается $n$ разными прожекторами.
Каждый прожектор освещает некую выпуклую фигуру, причем если выключить любой
один прожектор, то~арена будет по-прежнему полностью освещена, а~если выключить
любые два~--- будет освещена не~полностью.
При каких $n$ такое возможно?

\itemx{*}
Одиннадцати мудрецам завязывают глаза и~надевают каждому колпак одного из~1000
цветов.
После этого глаза развязывают, и~каждый видит все колпаки, кроме своего.
Затем одновременно каждый показывает остальным одну из~двух карточек~--- белую
или черную.
После этого все должны одновременно назвать цвет своих колпаков.
Как мудрецам заранее договориться, чтобы это удалось?
%Мудрецы могут заранее договориться о~своих действиях;
%мудрецам известно, каких 1000 цветов могут быть колпаки.

\itemx{*}
Есть $n$ разных ключей от~$n$ разных замков
(каждый ключ подходит ровно к~одной двери).
За~какое наименьшее число попыток можно гарантированно узнать, какую дверь
открывает какой ключ?

\end{problems}


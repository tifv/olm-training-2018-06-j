% $date: 2018-06-02
% $timetable:
%   g8r1:
%     2018-06-02:
%       1:
%   g8r2:
%     2018-06-02:
%       2:

\worksheet*{Индукция-2}

% $authors:
% - Андрей Борисович Меньщиков

\begin{problems}

\item
На~доске в~ряд написаны числа $1$ и~$1$.
Каждую минуту между соседними числами вписывается их сумма.
Какова сумма всех чисел через $n$~минут?

\item
Для всех натуральных $n \geq 3$ докажите, что
\(
    \frac{1}{n + 1} + \frac{1}{n + 2} + \ldots + \frac{1}{2 n}
>
    \frac{3}{5}
\).

\item
Докажите, что любое целое число единственным образом представимо в~виде суммы
некоторых из~чисел $1$, $-2$, $4$, \ldots, $(-2)^n$, \ldots

\item
Из~чисел от~$1$ до~$2 n$ выбраны $n + 1$ чисел.
Докажите, что среди выбранных чисел найдутся два, одно из~которых делится
на~другое.

\item
Докажите, что любая правильная дробь может быть представлена в~виде суммы
различных чисел, обратных к~целым.

\item
Докажите, что $f_{n}^2 + f_{n+1}^2 = f_{2n+1}$, где $f_{n}$~--- $n$-е число
Фибоначчи.

\item
При каких натуральных~$n$ число~$1$ можно представить в~виде суммы $n$ обратных
величин попарно различных натуральных чисел?

\item
Назовем число \emph{интересным}, если оно на~1 больше некоторого точного
квадрата.
Докажите, что произведение любых нескольких интересных чисел представимо в~виде
суммы квадратов двух целых чисел.

%\item
%Для того, чтобы проехать полный круг, машине требуется $100$ литров бензина.
%Эти $100$ литров распределены по~бензоколонкам вдоль трассы.
%Докажите, что есть точка, начав с~которой с~пустым баком, машина проедет весь
%круг.

%\item
%По~кругу стоит $2015$ чисел $\pm 1$.
%Известно, что количество $-1$ не~больше $671$.
%Докажите, что найдется число такое, что все частичные суммы, начинающиеся
%с~него (что вправо, что влево), положительны.

\item
Пусть $s(n)$~--- это количество разбиений натурального числа~$n$
в~упорядоченную сумму степеней двойки (например, $s(4) = 6$).
Найдите наименьшее $n > 2018$ такое, что $s(n)$ нечетно.

%\item
%Определим числа $K_{n}$:
%$K_{0} = 1$ и~$K_{n+1}= 1 + \min (2K_{[n/2]}, 3K_{[n/3]})$.
%Докажите, что $K_{n} \geq n$.

\item
Для всех натуральных $n$ докажите, что
\[
    \frac{1}{2} \cdot \frac{3}{4}
    \cdot \ldots \cdot
    \frac{2 n - 1}{2 n}
<
    \frac{1}{\sqrt{3 n}}
\, . \]

\end{problems}


% $date: 2018-06-01
% $timetable:
%   g8r1:
%     2018-06-01:
%       2:
%   g8r2:
%     2018-06-01:
%       1:

\worksheet*{Индукция}

% $authors:
% - Андрей Борисович Меньщиков

\begin{problems}

\item
На~полке стоит $55$~томов собрания сочинений В.\,И.\,Ленина.
За~раз разрешается взять несколько подряд идущих томов и~переставить
их~в~обратном порядке.
Докажите, что такими операциями можно расставить тома по~порядку.

%\item
%Петя умеет на~любом отрезке отмечать точки, которые делят этот отрезок пополам
%или в~отношении $n : (n + 1)$, где $n$~--- любое натуральное число.
%Петя утверждает, что этого достаточно, чтобы разделить отрезок на~любое
%количество одинаковых частей.
%Прав~ли он?

\item
Торт разрезали прямолинейными разрезами на~несколько кусков.
Оказалось, что одна сторона у~ножа была грязная.
Докажите, что найдется хотя~бы один чистый кусок.

\item
Плоскость разбита прямыми на~области.
Докажите, что области можно раскрасить в~два цвета так, чтобы граничащие
области были разного цвета.

\item
Выпуклый многоугольник разрезан непересекающимися диагоналями на~равнобедренные
треугольники.
Докажите, что в~этом многоугольнике найдутся две равные стороны.
% Эту задачу стоило поставить после прямоугольника 3 × n

\item
На~лестнице нарисованы стрелочки.
На~одной из~ступеней стоит человек.
Он идет со~ступеньки в~ту сторону, в~которую указывает стрелочка, после чего
стрелочка на~ступеньке, с~которой он сошел, обращается в~противоположную
сторону.
Докажите, что когда-нибудь человек покинет лестницу.

\item
В~прямоугольнике $3 \times n$ стоят фишки трех цветов, по~$n$ штук каждого
цвета.
Докажите, что можно переставить фишки в~каждой строке так, чтобы в~каждом
столбце были фишки всех цветов.

%\item
%Докажите, что любое целое число единственным образом представимо в~виде суммы
%некоторых из~чисел $1$, $-2$, $4$, \ldots, $(-2)^n$, \ldots

%\item
%Из~чисел от~$1$ до~$2 n$ выбраны $n + 1$ чисел.
%Докажите, что среди выбранных чисел найдутся два, одно из~которых делится
%на~другое.

\item
Несколько человек не~знакомы между собой.
Докажите, что можно познакомить некоторых из~них друг с~другом так, чтобы
ни~у~каких трех людей не~оказалось одинакового числа знакомых.

\item
При каких натуральных~$n$ квадрат можно разрезать на~$n$ меньших квадратов
(не~обязательно различных)?

%\item
%Докажите, что любая правильная дробь может быть представлена в~виде суммы
%обратных величин попарно различных целых чисел.

%\item
%При каких натуральных~$n$ число~$1$ можно представить в~виде суммы $n$ обратных
%величин попарно различных натуральных чисел?

%\item
%Для того, чтобы проехать полный круг, машине требуется $100$ литров бензина.
%Эти $100$ литров распределены по~бензоколонкам вдоль трассы.
%Докажите, что есть точка, начав с~которой с~пустым баком, машина проедет весь
%круг.

%\item
%По~кругу стоит $2015$ чисел $\pm 1$.
%Известно, что количество $-1$ не~больше $671$.
%Докажите, что найдется число такое, что все частичные суммы, начинающиеся
%с~него (что вправо, что влево), положительны.

\item
В~стране $n$ городов.
Между каждыми двумя из~них проложена либо автомобильная, либо железная дорога.
Турист хочет объехать страну, побывав в~каждом городе ровно один раз,
и~вернуться в~город, с~которого он начинал путешествие.
Докажите, что турист может выбрать город, с~которого он начнет путешествие,
и~маршрут так, что ему придется поменять вид транспорта не~более одного раза.

\item
Можно~ли отметить на~плоскости несколько точек так, чтобы на~расстоянии~$1$
от~каждой отмеченной точки находилось ровно $10$ отмеченных?

\end{problems}


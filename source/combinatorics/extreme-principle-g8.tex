% $date: 2018-06-03
% $timetable:
%   g8r1:
%     2018-06-03:
%       2:
%   g8r2:
%     2018-06-03:
%       1:

\worksheet*{Принцип крайнего}

% $authors:
% - Андрей Борисович Меньщиков

\begin{problems}

%Для обсуждения
%\item
%В~каждой клетке шахматной доски написано число, причем каждое число является
%средним арифметическим его соседей по~стороне.
%Докажите, что все числа равны.

\item
На~шахматной доске стоят несколько ладей.
Докажите, что найдется ладья, бьющая не~более двух других.

\item
Из~целых чисел от~$1$ до~$100$ удалили $k$ чисел.
Обязательно~ли среди оставшихся чисел можно выбрать $k$ различных чисел
с~суммой 100, если
\\
\subproblem $k = 9$;
\qquad
\subproblem $k = 8$?

\item
23 семиклассника бегают по~футбольному полю с~шишками в~руках.
По~свистку все они останавливаются, и~каждый кидает шишкой в~ближайшего к~нему
семиклассника (все расстояния между ними различны).
Докажите, что в~какого-то семиклассника шишка не~полетит.

\item
В~круге радиуса~1 отметили восемь точек.
Докажите, что расстояние между какими-то двумя из~них меньше~1.

\item
В~клетках таблицы $8 \times 8$ расставлены 64 различных целых числа.
Докажите, что найдутся две соседние по~стороне клетки, числа в~которых
отличаются хотя~бы на~5.

\item
7 грибников собрали вместе 100 грибов, причем никакие двое не~собрали
одинакового числа грибов.
Докажите, что какие-то трое грибников собрали не~меньше, чем остальные четверо.

\item
В~клетках таблицы $n \times n$ расставлены $n^2$ различных чисел.
В~каждой строке отметили наименьшее число, и~все отмеченные числа оказались
в~разных столбцах.
Затем в~каждом столбце отметили наименьшее число, и~все отмеченные числа
оказались в~разных строках.
Докажите, что оба раза отметили одни и~те~же числа.

\item
На~прямой выбрали $2 n + 1$ отрезков.
Известно, что любой из~них пересекается (хотя~бы по~точке) хотя~бы
с~$n$ другими.
Докажите, что найдется отрезок, пересекающийся со~всеми остальными.

\item
На~конгресс собрались ученые, среди которых есть друзья.
Оказалось, что каждые два из~них, имеющие на~конгрессе одинаковое число друзей,
не~имеют общих друзей.
Докажите, что найдется ученый, имеющий на~конгрессе ровно одного друга.

\item\emph{Теорема Сильвестра.}
На~плоскости даны $n$~точек.
Оказалось, что на~любой прямой, проходящей через любые две из~них, лежит еще
хотя~бы одна точка.
Докажите, что все точки лежат на~одной прямой.

\end{problems}


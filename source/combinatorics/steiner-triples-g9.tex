% $date: 2018-06-07
% $timetable:
%   g9r1:
%     2018-06-07:
%       1:
%   g9r2:
%     2018-06-08:
%       2:

\worksheet*{Тройки Штейнера и~другие системы подмножеств}

% $authors:
% - Александр Савельевич Штерн

\begingroup
    \def\binom#1#2{\mathrm{C}_{#1}^{#2}}%

\subsection*{Обсуждение}

\emph{Антицепь} из~подмножеств~--- это набор подмножеств, никакие два
из~которых друг в~друга не~вложены.

\claim{Теорема Шпернера об~антицепях}
Самая большая антицепь в~$n$-элементном множестве содержит
$\binom{n}{[n/2]}$ подмножеств.

\begin{exercises}

\item
Сколько полных цепей существует в~$n$-элементном множестве?

\item
Подмножество~$A$ множества~$X$ содержит $k$~элементов.
Сколько существует полных цепей, содержащих подмножество~$A$?

\item
В~$n$-элементном множестве выделили несколько подмножеств, содержащих
$x_{1}$, $x_{2}$, \ldots, $x_{k}$ элементов соответственно.
Известно, что ни~одно из~этих множеств не~содержится в~другом.
Докажите неравенство
\[
    \frac{1}{\binom{n}{x_{1}}} + \frac{1}{\binom{n}{x_{2}}}
    + \ldots +
    \frac{1}{\binom{n}{x_{n}}}
\leq
    1
\, . \]

\item
Докажите теорему Шпернера.

\end{exercises}


\subsection*{Задачи}

\begin{problems}

\item\jeolmlabel{/combinatorics/steiner-g9/:problem:example}%
Комиссия состоит из~100 человек.
На~каждое заседание приходит три члена комиссии.
Может~ли через некоторое время получиться так, что любые два члена комиссии
встретятся на~заседаниях ровно один раз?
Тот~же вопрос, если в~комиссии 101 человек.

\end{problems}

\claim{Определение}
Пусть дано некоторое конечное множество.
\emph{Системой троек Штейнера} этого множества называется такой набор его
трехэлементных подмножеств, что любая пара элементов этого множества
встречается ровно в~одном подмножестве.

\begin{problems}

\item
Пусть для некоторого $n$-элементного множества удалось построить
систему троек Штейнера.
Анализируя решения
задачи~\jeolmref{/combinatorics/steiner-g9/:problem:example},
установите необходимое условие на~число~$n$.

\item
Построить систему троек Штейнера для
7-эле\-мен\-тного множества,
21-эле\-мен\-тного множества,
49-эле\-мен\-тного множества.

\item
Известно, что система троек Штейнера существует для $m$-элементного и~для
$n$-элементного множества.
Доказать, что она существует и~для множества, содержащего $mn$~элементов.

\item
В~множестве, состоящем из~$n$ элементов, выбрано $2^{n-1}$ подмножеств,
каждые три из~которых имеют общий элемент.
Докажите, что все эти подмножества имеют общий элемент.

\item
В~классе 16 учеников.
Каждый месяц учитель делит класс на~две группы.
Какое наименьшее количество месяцев должно пройти, чтобы каждые два ученика
в~какой-то из~месяцев оказались в~разных группах?

\item
Множество~$A$ содержит 101 число, не~превосходящее миллиона.
Сдвигом множества~$A$ называется множество чисел, каждое из~которых получается
прибавлением ко~всем его элементам некоторого фиксированного числа, также
не~превосходящего миллиона.
Докажите, что найдется сто попарно не~пересекающихся сдвига множества~$A$.

\end{problems}

\endgroup

